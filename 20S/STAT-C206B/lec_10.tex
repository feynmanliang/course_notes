\lecture{10}{2020-02-20}{}

Let $\xi \in S$ and $\eta \in T$, $\xi \deq \tilde{\xi}$.

$\mu$ the regular conditional distribution of $\eta$ given $\xi$.

$(\xi, \eta) \deq (\tilde{\xi}, \tilde{\eta})$.

\begin{theorem}[Fubini/Tonelli]
\end{theorem}

\todo{}
\begin{corollary}[extended Minkowski inequality]
  Let $\mu$, $\nu$, and $f$ be such as in the previous theorem ??
  and assume
  $\mu f(t) = \int f(s,t) \mu(ds)$ exists for $t \in T$.
  Write $\|f\|_p(s) = (\nu \mid f(s, \cdot)\rvert^p)^{1/p}$.
  Then
  \[
    \|\mu f\|_p \leq \mu \|f\|_p, \quad p \geq 1
  \]
\end{corollary}

\begin{proof}
  Follows from H\"older's inequality in the usual way.
\end{proof}

Recall the regular Minkowski inequality: for $g_k : T \to \bR$ and
$\alpha_k \geq 0$ for $k \in [n]$,
\begin{align*}
  \| \sum^{n}_{i=1} \alpha_i g_i \|_p \leq \sum^{n}_{i=1} \alpha_i \|g_i\|_p
\end{align*}
This is a special case of the corollary by taking $S = [n]$,
$\mu(\{k\}) = \alpha_k$, and $f(k,t) = g_k(t)$.

\subsection{Ergodic theory}
\label{sub:Ergodic theory}

Fix measure space $(S, \cS, \mu)$

\begin{definition}
  A transformation $T$ on $S$ is \emph{$\mu$-preserving} or
  \emph{measure-preserving} if $\mu \circ T^{-1} = \mu$.

  Equivalently, if $\xi$ is a random element of $S$ with distribution $\mu$,
  $T$ is measure-preserving iff $T \xi \deq \xi$.
\end{definition}

\begin{example}
  For a random sequence $\xi = (\xi_i)_i$,
  let $\theta$ denote the left shift operator
  $\theta (\xi_i)_i = (\xi_{i+1})_i$.
  $\xi$ is \emph{stationary} if $\theta \xi \deq \xi$,
  i.e. $\theta$ is measure-preserving for the distribution of $\xi$.
\end{example}

\begin{lemma}[Statinarity and invariance]
  For random $\xi \in S$ and measurable $T$ on $S$, $T \xi \deq \xi$
  iff $(T^n \xi)$ is stationary, in which case $(f \circ T^n \xi)$ is
  stationary for every measurable $f$.

  Conversely, any stationary random sequence admits such a representation.
\end{lemma}

\begin{proof}
  Assuming $T \xi \deq \xi$,
  \[
    \theta(f \circ T^n \xi)
    = (f \circ T^{n+1} \xi)
    = (f \circ T^n T \xi)
    \deq (f \circ T^n \xi)
  \]
  Conversely, if $\eta = (\eta_i)$ is stationary then
  $\eta_n = \pi_0(\theta^n \eta)$ with $\pi_0(x_0, x_1, \ldots) = x_0$.
  $\theta \eta \deq \eta$ by stationarity of $\eta$.
\end{proof}

Let $\cS^\mu$ denote the $\mu$-completion of $\cS$.

\begin{definition}
  A set $I \subset S$ is \emph{invariant} if $T^{-1} I = I$ and
  \emph{almost invariant} if $T^{-1} I = I$ $\mu$-a.e. in the sense
  that $\mu(T^{-1} I \Delta I) = 0$.

  Since inverse images preserve set operations, the classes
  $\cI$ and $\cI'$  of invariant sets in $\cS$ and almost invariant
  sets in $\cS^\mu$ form $\sigma$-fields in $S$, called the
  \emph{invariant} and \emph{almost invariant} $\sigma$-fields.

  $f$ measurable function on $S$ is \emph{invariant} if $f \circ T \equiv f$
  and \emph{almost invariant} if $f \circ T = f$ $\mu$-a.e..
\end{definition}

\begin{example}
  $S = \{0,1\}^\infty$, $T = \theta$, then
  $I = \{ (x_i)_i : x_k = 0 \text{ i.o. }\}$
  is an invariant set.
\end{example}

We are heading to the ergodic theorem; if $\xi = (\xi_i)_i$ is
stationary $S$-valued sequence, $f : S \to \bR_+$ $\cS$-measurable, then
$\frac{1}{denom} \sum_{0 \leq k \leq n} f(\xi_k)$ converges almost surely.
This is a generalization of the law of large numbers, where this
limit converges to the constant $f(\xi_0)$.

In general, the limit may be random rather than a constant.
\begin{example}
  Let $\xi$ have distribution
  $\frac{1}{2}\delta_{(0,0,\ldots)} + \frac{1}{2} \delta_{(1,1,\ldots)}$.
  Then the limit is random.
\end{example}

\begin{lemma}
  For measure $\mu$ and measurable transform $T$ on $S$, let
  $f : S \to S'$ be measurable mapping into Borel space $S'$.

  Then $f$ is invariant or almost invariant iff 
  it is $\cI$-measurable or $\cI'$-measurable respectively.
\end{lemma}

\begin{proof}
  TODO
\end{proof}

Write $\cI^\mu$ for the $\mu$-completion of $\cI$ in $\cS^\mu$, the
$\sigma$-field generated by $\cI$ and $\mu$-null sets in $\cS^\mu$.

\begin{theorem}
  For any distribution $\mu$ and $\mu$-preserving transform $T$
  on $S$, the associated invariant and almost invariant $\sigma$-fields
  $\cI$ and $\cI'$ respectively are related by $\cI' = \cI^\mu$.
\end{theorem}

$\mu(A \Delta B)$ is a pseudometric on sets.

\begin{proof}
  If $J \in \cI^\mu$, then $\mu(I \Delta J) = 0$ for some $I \in \cI$.
  Since $T$ is $\mu$-preserving, we get ???
\end{proof}

\begin{definition}
  $T$ is \emph{ergodic} for $\mu$ or simply \emph{$\mu$-ergodic} if the
  invariant $\sigma$-field $\cI$ is $\mu$-trivial in the sense
  $\mu I \in \{0,1\}$ for every $I \in \cI$.

  Depending on viewpoint, we may also say $\mu$ is ergodic for $T$.

  Randomm $\xi$ is ergodic if $\Pr[\xi \in I] \in \{0,1\}$ for any $I \in \cI$,
  i.e. $\cI_\xi = \xi^{-1} I$ in $\Omega$ is $\Pr$-trivial.
\end{definition}

\begin{lemma}
  Let $\xi \in S$ be random with distribution $\mu$, $T$ a $\mu$-preserving
  mapping on $S$. Then $\xi$ is $T$-ergodic iff $(T^n \xi)$ is $\theta$-ergodic
  in which case $(f \circ T^n \xi)$ is $\theta$-ergodic for every measurable
  $f$.
\end{lemma}

Write $\cI_\xi = \xi^{-1} \cI$.

\begin{theorem}[Ergodic theorem, Birkhoff]
  Let $\xi$ be a random element in $S$ with distribution $\mu$, $T$ a
  $\mu$-preserving map on $S$ with invariant $\sigma$-field $\cI$.
  Then for any measurable $f \geq 0$ on $S$
  \[
    n^{-1} \sum_{0 \leq k < n} f(T^k \xi) \asto \bE[f(\xi) \mid \cI_\xi]
  \]
  The same convergence holds in $L^p$ for some $p \geq 1$ when 
  $f \in L^p(\mu)$.
\end{theorem}

The LHS is a limit of Cesaro means.

\begin{definition}
  Distributions $\mu_n$ on $\bN$ are \emph{asymptotically invariant}
  if $\|\mu_n - \mu_n \ast \delta_s \| \to 0$ for every $s \in \bN$
  where $\|\cdot\|$ denotes the total variation norm, i.e.
  \[
    \|\mu_n - \mu_n \ast \delta_s\|
    = \sum_k \lvert \mu_n(\{k\}) - \mu_n \ast \delta_s(\{k\}) \rvert
  \]
  and $\ast$ denotes convolution, i.e.
  \[
    \mu_n \ast \delta(\{k\{) = \begin{cases}
          \mu_n(k-s), &\text{ if }k \geq s\\
          0, &\text{ ow}
    \end{cases}
  \]
\end{definition}

\begin{example}
  $\mu_n = n^{-1} \sum_{0 \leq k < n} \delta_k$.
\end{example}

\begin{corollary}[Extended mean ergodic theorem]
  For any $p \geq 1$, consider on $\bN$ a stationary
  $L^p$-valued process $X$ and some asymptotically invariant $\mu$.
  Then
  \[
    \mu_n X = \int_\bN X_s \mu_n(ds) = \sum_{k in \bN} X_k \mu_n(\{k\})
    \to \bar{X} \coloneqq \bE[X \mid \cI_X]
  \]
  in $L^p$.
\end{corollary}

\begin{proof}
  Let $\nu_m = m^{-1} \sum_{0 \leq k < m} \delta_k$
  so $\nu_m X \to \bar{X}$ in $L^p$ by mean ergodic theorem.
  By generalized Minkowski's inequality along with stationarity of $X$,
  invariance of $\bar{X}$ and dominated convergence, we get as $n \to \infty$
  and then $m \to \infty$
  \begin{align*}
    \|\mu_n X - \bar{X}\|_p
    &\leq \|\mu_n X - (\mu_n \ast \nu_m) X \|_p
    + \|(\mu_n \ast \nu_m) X - \bar{X}\|_p \\
    &\leq \|\mu_n - \mu_n \ast \nu_m\| \|X_0\|_p
    + \int \|(\delta_s \ast \nu_m) X - \bar{X}\|_p \mu_n(ds) \\
    &\leq \|X\|_p \int \|\mu_n - \mu_n \ast \delta_t\| \nu_m(dt)
    + \|\nu_m X - \bar{X}\|_p \to 0
  \end{align*}
\end{proof}
