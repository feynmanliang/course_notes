\lecture{2}{2020-01-23}{Measure theory continued}

Today we will start off by completing the construction of the
Haar measure. All metric spaces are Hausdorff.

Recall that \myref{thm:kakutani} requires equicontinuity of a set of continuous
maps. The following result gives an alternative characterization of
equicontinuity.

\begin{theorem}[Generalization of Arzela-Ascoli]
  \label{thm:arzela-ascoli}
  Let $X$ be a compact topological space.
  A subset of function $F \subset \cC(X)$ is relatively compact in topology
  induced by uniform norm $\iff$ $F$ is equicontinuous and pointwise
  bounded.
\end{theorem}

$f \mapsto \prescript{}{s}{f}$ linear transform of $\cC(G)$,
$\|\prescript{}{s}{f} - \prescript{}{s}{g}\| = \|f - g\|$,
$G$ acts as a group of isometries on $\cC(G)$. In particular,
this group is equicontinuous.

\begin{proof}[Proof of \cref{thm:haar-measure}]
  Fix $f \in \cC(G)$. Let $\cC_f$ denote the convex hull of all left
  translates of $f$. Elements $g \in \cC_f$ are finite sums of form
  \[
    g(x) = \sum_{\mathrm{finite}} a_i f(s_i x), \qquad a_i > 0, \sum_{\mathrm{finite}} a_i = 1
  \]
  Clearly $\|g\|_\infty = \max\{\lvert g(x)\rvert : x \in G\} \leq \|f\|_\infty$, thus
  sets of the form $\cC_f(x) = \{g(x) : g \in \cC_f\}$ are bounded.

  Since $G$ is compact, $f$ is uniformly continuous hence for $\eps > 0$
  there exists a neighborhood $V = V_\eps$ of the identity $e \in G$ such that
  \[
    y^{-1} x \in V \implies \lvert f(x) - f(y) \rvert \leq \eps
  \]
  Since $(s^{-1} y)^{-1} s^{-1} x = y^{-1} x$, we also have \todo{}.

  By \myref{thm:arzela-ascoli},
  $\cC_f$ relatively compact in $\cC(G)$, so its
  closure $K_f = \overline{\cC_f}$ is compact convex.
  $G$ acts by left translation on $\cC(G)$ and leaves $\cC_f$ invariant,
  hence $K_f$ invariant as well. By \myref{thm:kakutani}, there is a fixed
  point $g \in K_f$ of this action of $G$ on $K_f$ which satisfies
  \[
    \pre{s}{g} = g \; (\forall s \in G)
    \quad\implies\quad
    g(s^{-1}) = \pre{s}g(e) = g(e) = c\; (\forall s \in G)
  \]
  for some constant $c$.

  By definition of $K_f$, for any $\eps > 0$ there is
  $\{s_1, \ldots, s_n\} \subset G$ and $a_i > 0$ such that
  \[
    \sum_{i=1}^n a_i = 1,
    \quad \text{and}\quad \left\lvert c - \sum_{i=1}^n a_i f(s_i x) \right\rvert < \eps
    \qquad (\forall x \in G)
  \]
  We first show there is only one constant function in $K_f$, so the fix
  point $g \in K_f$ is unique.

  Start with same construction as before expect now use right translations of
  $f$ (i.e. using the opposite group $G'$ of $G$, or the function $f' =
  f(x^{-1})$, obtaining relatively compact set $\cC_f'$ with compact convex
  closure $K_f'$ containing constant function $c'$).

  \begin{note}{Opposite group}
    The opposite group $g'$ of the group $G$ is the group that
    coincides with $G$ as a set but has group operation
    $(x,y) \mapsto y^{-1} x^{-1}$
  \end{note}
  It will be enough to show $c=c'$, since all constants $c \in K_f$ must be
  equal to one chosen constant $c' \in K_f'$ and conversely.

  By construction, exists finite combination of right translates close
  to $c'$ i.e.
  \begin{align}
    \label{eq:combo-close-to-constant}
    \lvert c' - \sum_j b_j f(x t_j) \rvert < \eps
    \qquad(\text{for some $t_j \in G$, $b_j > 0$ with $\sum_j b_j = 1$})
  \end{align}
  Multiply by $a_i$ and put $x = s_i$ to get
  \[
    \lvert c' a_i - \sum_j a_i b_j f(s_i t_j) \rvert < \eps a_i
  \]
  Summing over $i$
  \[
    \lvert c' - \sum_{i, j} a_i b_j f(s_i t_j) \rvert
    = \lvert c' \sum_i a_i - \sum_{i,j} a_i b_j f(s_i t_j) \rvert
    < \eps \sum_i a_i
    = \eps
  \]
  Operating symmetrically on \cref{eq:combo-close-to-constant}
  (mult by $b_j$, $x = t_j$) \todo{finish}

  The following properties are obvious:
  \begin{itemize}
    \item $m(\ind)= 1$ since $K_f = \{1\}$ for $f=1$
    \item $m(f) \geq 0$ if $f \geq 0$
    \item $m(a f) = a m(f)$ for any $a \in \bR$ (since $K_{af} = K_f$)
    \item $m(\pre{s}{f}) = m(f) = m(f_s)$ (by uniqueness)
  \end{itemize}
  It remains to show that $m$ is additive (hence linear). Take $f,g \in \cC(G)$
  and start with \cref{eq:combo-close-to-constant} with $c = m(f)$, further
  letting $h(x) = \sum_i a_i g(s_i x)$. Since $h \in \cC_g$, we have $\cC_h
  \subset \cC_g$ hence $K_h \subset K_g$. But $K_g$ contains only one constant
  so in fact $m(h) = m(g)$.

  We can write
  \[
    \lvert m(h) - \sum b_j h(t_j x) \rvert < \eps
  \]
  for finitely many $t_j \in G$ and $b_j > 0$, $\sum_j b_j = 1$.

  Using this definition of $h$ and $m(h) = m(g)$ gives
  \[
    \lvert m(g) - \sum_{i,j} a_i b_j g(s_i t_j x) \rvert < \eps
  \]
  However, multiplying \cref{eq:combo-close-to-constant} by $b_j$
  and replacing $x$ \todo{}, adding (9) and (10) gives
  \[
    \lvert m(f) + m(g) - \sum_{i,j} a_i b_j (f + g) (s_i t_j x) \rvert < 2 \eps
  \]
  Thus $m(f) + m(g) \in K_{f+g}$, establishing additivity.
  Note that the only constant in $K_{f+g}$ is \todo{???}.
\end{proof}

We now want to head towards some integration against probability measures.

\begin{definition}
  A topological space $X$ is \emph{normal} if for any disjoint closed
  sets $Y$ and $Z$ there exists disjoint open sets $U$ and $V$
  such that $Y \subset U$ and $Z \subset V$.
\end{definition}

\begin{figure}[ht]
  \centering
  \incfig[0.5]{normal-topological-space}
  \caption{Normal topological spaces admit separating closed
  sets with two disjoint open sets}
  \label{fig:normal-topological-space}
\end{figure}

\begin{definition}
  $X$ is \emph{completely regular} if for all $y \in X$ and
  every closed $Z \subset X \setminus \{y\}$ there exists
  $f : X \to [0,1]$ continuous such that $f(y) = 0$ and $f(z) = 1$
  for all $z in Z$.
\end{definition}

\begin{lemma}[Urysohn]
  Every normal space is completely regular.
\end{lemma}

\begin{definition}
  $m$ a set function on $X$ with values in $[0,+\infty]$
  such that $m(\emptyset) = 0$.
  $A \subset X$ is \emph{Carath\'eodory measurable} wrt
  $m$ (Carath\'eodory $m$-measurable) if, for every $E \subset X$
  we have
  \[
    m(E \cap A) + m(E \setminus A) = m(E)
  \]
  Let $\fM_m$ denote class of all Carath\'eodory $m$-measurable sets.
\end{definition}

\begin{theorem}
  \begin{enumerate}
    \item $\fM_m$ is an algebra, $m$ is additive on $\fM_m$
    \item For all sequences of pairwise disjoint $A_i \in \fM_m$: ???
    \item If $m$ is an outer measure on $X$, then $\fM_m$ is a $\sigma$-algebra,
      $m$ is countably additive on $\fM_m$, and
      $m$ is complete on $\fM_m$
  \end{enumerate}
\end{theorem}

\begin{example}
  Let $\fX$ be a family of subsets of $X$ such that $\emptyset \in \fX$.
  Given $\tau : \fX \to [0,+\infty]$ with $\tau(\emptyset) = 0$, set
  \[
    \fm(A) = \inf\left\{
      \sum_{n=1}^\infty \tau(X_n) : X_n \in \fX, A \subset \cup_{n=1}^\infty X_n
    \right\}
  \]
  where $\fM(A) = \infty$ in the absence of such sets $X_n$.
  Then $\fm$ is an outer measure, denoted $\tau^*$.
\end{example}

The \emph{Borel $\sigma$-algebra}, $\cB(X)$, is generated by all
open sets.

Let $\cB a(x)$ denote the \emph{Baire $\sigma$-algebra}, generated by
\begin{align}
  \label{eq:functionally-open}
  \{x \in X : f(x) > 0 \}
\end{align}
where $f$ is a continuous function on $X$.
It is the smallest $\sigma$-algebra where all continuous
functions are measurable.

The sets of the form \cref{eq:functionally-open} are called
\emph{functionally open}.

\begin{lemma}
  In a metric space $X$, any closed set is the set of zeros of a continuous
  function. Hence, $\cB(X) = \cB a(X)$.
\end{lemma}

\begin{lemma}[Baire sets are countably determined]
  Every Baire set is determined by some countable
  family of functions, i.e. has the form
  \[
    \{x : (f_i(x))_i : i < \infty, f_i \in \cC(X) \}
  \]
\end{lemma}

A consequence of the monotone class theorem ???

Throughout, we consider (signed) measures of \emph{bounded variation}
unless explicitly denoted otherwise.

\begin{definition}
  Let $X$ be a topological space.
  \begin{itemize}
    \item A countably additive measure on $\cB(X)$ is called a
      \emph{Borel measure}
    \item A countably additive measure on $\cB a(X)$ is called a
      \emph{Baire measure}
    \item A Borel measure $\mu$ on $X$ is called \emph{Radon measure}
      if every $B \in \cB(X)$ can be approximated from the inside
      by compact sets: for $\eps > 0$ exists $K_\eps \subset B$
      such that $\lvert \mu \rvert(B \setminus K_\eps) < \eps$.
  \end{itemize}
\end{definition}

When are two Borel measures equal?
\begin{lemma}
  If two Borel measures coincide on all open sets, thne they
  coincide on all Borel sets.
\end{lemma}

\begin{proof}
  Split $\mu = \mu^+ - \mu^-$ and notice that each of the two
  components are nonnegative and coincide on open sets.
  By monotone class theorem, $\mu^+ = \mu^-$. \todo{finish}
\end{proof}

$\mu$ is Radon iff $\lvert \mu \rvert$ is Radon iff both $\mu^+$
and $\mu^-$ are Radon.

Inner and outer approximatino of measures on $\bR^n$:

\begin{theorem}
  $\mu \geq 0$ on $\cB(\bR^n)$, then any Borel set $B \subset \bR^n$ and
  any $\eps > 0$ exists $U_\eps$ open and $F_\eps$ closed such that
  $F_\eps \subset A \subset U_\eps$ and $\mu(U_\eps \setminus F_\eps) < \eps$.
\end{theorem}

\begin{proof}
  Let $\cA$ the class of all sets $A in \cB$ such that
  $F_\eps \subset A \subset U_e$ and $\mu(U_\eps \setminus F_\eps) < \eps$
  for some closed set $F_\eps$ and open set $U_\eps$.

  Every closed $A$ is in $\cA$, since we can take
  $F_\eps = A$ and $U_\eps$ some open $\delta$-nbd
  and consider $\delta \to 0$.

  It suffices to show that $\cA$ is a $\sigma$-algebra, since the closed
  sets generate $\cB$. $\cA$ is closed wrt complements, so it remains
  to verify closure under countable union.

  Let $A_j \in \cA$, $\eps > 0$. Then exists closed $F_j$
  and open $U-j$ such that $F_j \subset A_j \subset U-j$ and
  $\mu(U_j \setminus F_j) < \eps 2^{-j}$ for $j \in \bN$.

  The set $U = U_{j=1}^\infty U_j$ is open, and $Z_k = U_{j=1}^k F_j$ is closed.

  Observe $Z_k \subset \cup_{j=1}^\infty A_j \subset U$ and for
  sufficiently large $k$ $\mu(U \setminus Z_k) < \eps$.

  Indeed, $\mu(\cup_j^\infty \mu_j \setminus F_j) < 2^{}$ \todo{??}
\end{proof}

\begin{definition}
  Set function $\mu \geq 0$ defined on $\cA \subset 2^X$
  is \emph{tight} on $\cA$ if $\forall \eps > 0$ exists compact
  $K_\eps \subset X$ such that $\mu(A) < \eps$ for all $A \in \cA$
  that does not meet $K_\eps$.

  Additive set function $\mu$ of bounded variation on an algebra is
  \emph{tight} if its total variation $\lvert \mu \rvert$ is tight.
\end{definition}

A Borel measure is tight iff $\forall \eps > 0$ exists compact
$K_\eps$ such that $\lvert \mu \rvert(X \setminus K_\eps) < \eps$
(the ``total variation measure'').

The second definition is necessary to handle Baire sets.

\begin{definition}
  $\mu$ is \emph{regular} if $\forall A \in \cA, \eps > 0$, $\exists F_\eps$
  closed such that $F_\eps \subset A$, $A \setminus F_\eps \in \cA$,
  and \todo{??}
\end{definition}

Theorem 27 implies any Borel measure on $\bR_n$ is regular, and the
same proof works to show any Borel measure on metric space is regular.

\begin{corollary}[Baire measures are regular]
  Every Baire measure $\mu$ on topological space $X$ is regular.
  Moreover, for every Baire set $E$ and $\eps > 0$,
  there exists a continuous function $f$ on $X$ such that
  $f^{-1}(0) \subset E$ and $\lvert \mu \rvert(E \setminus f^{_1}(0)) < \eps$.
\end{corollary}
