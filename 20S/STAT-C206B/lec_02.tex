\lecture{2}{2020-01-23}{Measure theory}

Throughout our discussion, all topological spaces are assumed Hausdorff
unless explicitly noted otherwise.

\subsection{Construction of Haar measure}

\begin{definition}
  A \emph{topological group} is a group equipped with a topology
  such that the group operations $(g, h) \mapsto gh$ and $g \mapsto g^{-1}$
  are continuous.
\end{definition}

\begin{theorem}[Existence of Haar Measure]
  \label{thm:haar-measure}
  Let $G$ be a compact topological group and $\cC(G)$ the set of continuous
  maps $G \to \bR$.
  Then there is a unique linear form $m : \cC(G) \to \bR$ such that
  \begin{enumerate}
    \item $m(f) \geq 0$ for $f \geq 0$ (positive)
    \item $m(\ind) = 1$ (normalized)
    \item $m(\prescript{}{s}{f}) = m(f)$ where
      $\prescript{}{s}{f}(g) = f(s^{-1} g)$
      for $s,g \in G$ (left invariant)
    \item $m(f_s) = m(f)$ where $f_s(g) = f(gs)$ (right invariant)
  \end{enumerate}
  $m$ is called the \emph{Haar measure} on $G$.
\end{theorem}

We will need the following theorem to relate compactness
with equicontinuity:
\begin{theorem}[Generalization of Arzela-Ascoli]
  \label{thm:arzela-ascoli}
  Let $X$ be a compact Hausdorff space.
  A subset of $\bR$-valued continuous functions $F \subset \cC(X)$
  is relatively compact in topology
  induced by uniform norm $\|\cdot\|_\infty$
  $\iff$ $F$ is equicontinuous and pointwise bounded.
\end{theorem}

\begin{proof}[Proof of \cref{thm:haar-measure}]
  Fix $f \in \cC(G)$ and let $\cC_f$ denote the convex hull of all left
  translates of $f$, i.e. $g \in \cC_f$ are finite sums of form
  \[
    g(x) = \sum_{\mathrm{finite}} a_i f(s_i x), \qquad a_i > 0, \sum_{\mathrm{finite}} a_i = 1, s_i \in G
  \]
  Clearly $\|g\|_\infty \leq \|f\|_\infty < \infty$, thus $\cC_f(x) = \{g(x) : g \in
  \cC_f\}$ is bounded for all $x \in G$ hence $\cC_f$ is pointwise bounded.

  As $f$ is a continuous function on compact $G$, it is uniformly continuous
  hence for $\eps > 0$ there exists a neighborhood $V_\eps$ of the identity
  $e \in G$ such that
  \[
    y^{-1} x \in V_\eps \implies \lvert f(x) - f(y) \rvert \leq \eps
  \]
  Since $(s^{-1} y)^{-1} s^{-1} x = y^{-1} x$, we also have
  \[
    y^{-1} x \in V_\eps \implies \lvert \pre{s}{f}(y) - \pre{s}f(x) \rvert < \eps
  \]
  Since $g \in \cC_f$ are convex combinations of $\pre{s}{f}$,
  by the triangle inequality
  \[
    y^{-1} x \in V_\eps \implies \lvert g(y) - g(x) \rvert < \eps
  \]
  As this works for any $g \in \cC_f$, we have that $\cC_f$ is equicontinuous.

  By \myref{thm:arzela-ascoli}, $\cC_f$ is relatively compact in $\cC(G)$,
  so its closure $K_f \coloneqq \overline{\cC_f}$ is compact (and still convex).

  Consider $G$ acting on $\cC(G)$ by left translation $f \mapsto
  \prescript{}{s}{f}$. Notice $G \cC_f \subset \cC_f$ (as $\cC_f$ already
  contains all finite convex combinations of all left translations of $f$) and
  hence $G K_f \subset K_f$ as well.

  Furthermore,
  $\|\prescript{}{s}{f} - \prescript{}{s}{g}\|_\infty = \|f - g\|_\infty$
  so $G$ acts as a group of isometries on $\cC(G)$.
  In particular, this group is equicontinuous (with the same $U = V$
  in \cref{def:equicontinuous}).

  Taking $\fG = G$ and $K = K_f$ in \myref{thm:kakutani}, there is a fixed
  point $g \in K_f$ of this action of $G$ on $K_f$ which satisfies
  \[
    \pre{s}{g} = g \; (\forall s \in G)
    \quad\implies\quad
    g(s^{-1}) = \pre{s}g(e) = g(e) = c\; (\forall s \in G)
  \]
  for some constant $c \in \bR$ (which we will later use to
  define $m(f) \coloneqq c$).

  We first show there is only one constant function in $K_f$, so the fix
  point $G g = \{g\} = \{c \ind\}$ is unique and $m(f) = c$ is well defined.
  For any constant function $c \ind \in K_f$ and $\eps > 0$, we can
  (because $K_f = \overline{\cC_f}$) find $\{s_1, \ldots, s_n\} \subset G$
  and $a_i > 0$ such that
  \begin{align}
    \label{eq:combo-close-to-constant}
    \sum_{i=1}^n a_i = 1,
    \quad \text{and}\quad \left\lvert c - \sum_{i=1}^n a_i f(s_i x) \right\rvert < \eps
    \qquad (\forall x \in G)
  \end{align}
  for any $\eps > 0$.

  Similarly, consider the same construction as before expect now use right
  translations of $f$ (i.e. using the opposite group $G'$ of $G$, or the function $f' =
  f(x^{-1})$, obtaining relatively compact set $\cC_f'$ with compact convex
  closure $K_f'$ with fix point $g' = c' \ind$).
  Approximating $c' \ind$ using $\cC_f'$, we have
  \begin{align}
    \left\lvert c' - \sum_j b_j f(x t_j) \right\rvert < \eps
    \qquad(\text{for some $t_j \in G$, $b_j > 0$ with $\sum_j b_j = 1$})
  \end{align}
  \begin{note}{Opposite group}
    The opposite group $g'$ of the group $G$ is the group that
    coincides with $G$ as a set but has group operation
    $(x,y) \mapsto y^{-1} x^{-1}$
  \end{note}
  Summing over $i$
  \[
    \left\lvert c' - \sum_{i, j} a_i b_j f(s_i t_j) \right\rvert
    < \eps \sum_i a_i
    = \eps
  \]
  Operating symmetrically on \cref{eq:combo-close-to-constant}
  (multiply by $b_i$ and put $x = t_i$) shows
  \[
    \left\lvert c - \sum_{i, j} a_i b_j f(s_i t_j) \right\rvert
    < \eps
  \]
  Together, we have $\lvert c' - c \rvert < 2 \eps$ so taking $\eps \to 0$
  shows $c' = c$. Since $c \ind \in K_f$ was an arbitrary constant function,
  we have that the constant function in $K_f$ is actually unique and so
  the function $m(f) \coloneqq c \in K_f$ is well defined. Moreover,
  $m(f) \ind$ is the \emph{only} constant function which can be arbitrary
  well approximated by convex combinations of left or right translates
  of $f$.

  The following properties are obvious:
  \begin{itemize}
    \item $m(\ind)= 1$ since $K_f = \{1\}$ for $f=\ind$
    \item $m(f) \geq 0$ if $f \geq 0$
    \item $m(\pre{s}{f}) = m(f) = m(f_s)$ (since $K_{\pre{s}{f}} = K_f$,
      $K'_f = K'_{f_s}$, and uniqueness of $m(f) \ind$ being the only
      constant function approximable by both $K_f$ and $K_f'$)
    \item $m(a f) = a m(f)$ for any $a \in \bR$ (since $K_{af} = K_f$)
  \end{itemize}

  To show $m$ is linear, it suffices (due to the last bullet above) to show
  that $m$ is additive. Fix $f,g \in \cC(G)$. Approximate $m(f)$
  using $K_f$ to get
  \begin{align}
    \label{eq:approx-mf-using-Kf}
    \left\lvert m(f) - \sum_{\mathrm{finite}} a_i f(s_i x) \right\rvert
  \end{align}
  Define $h(x) = \sum_{\mathrm{finite}} a_i g(s_i x)$ using the same $a_i$
  and $s_i$ and approximate $m(h)$ using $\cC_h$ to get
  \[
    \left\lvert m(h) - \sum_{\mathrm{finite}} b_j h(t_j x) \right\rvert < \eps
  \]
  Since $h \in \cC_g$, we have $\cC_h
  \subset \cC_g$ hence $K_h \subset K_g$. But $m(g) \ind \in K_g$ is the only
  constant function so $m(h) = m(g)$ and (after expanding the definition of $h$)
  we have
  \[
    \left\lvert m(g) - \sum_{i,j < \infty} a_i b_j g(s_i t_j x) \right\rvert < \eps
  \]
  On the other hand, multiplying \cref{eq:approx-mf-using-Kf} by $b_j$
  replacing $x$ with $t_j x$, summing over $j$, and finally adding with
  the above ineqeuality gives
  \[
    \lvert m(f) + m(g) - \sum_{i,j} a_i b_j (f + g) (s_i t_j x) \rvert < 2 \eps
  \]
  Thus $m(f) + m(g) \in K_{f+g}$, establishing additivity.
  Note that the only constant in $K_{f+g}$ is $(m(f) + m(g))\ind$.
\end{proof}

\subsection{Facts from topology}

We now want to head towards some integration against probability measures
defined on spaces more abstract than $\bR^n$.


\begin{definition}
  A topological space $X$ is \emph{normal} if for any disjoint closed
  sets $Y$ and $Z$ there exists disjoint open sets $U$ and $V$
  such that $Y \subset U$ and $Z \subset V$.
\end{definition}

\begin{figure}[ht]
  \centering
  \incfig[0.5]{normal-topological-space}
  \caption{Normal topological spaces admit separating closed
  sets with two disjoint open sets}
  \label{fig:normal-topological-space}
\end{figure}

\begin{definition}
  $X$ is \emph{completely regular} (\emph{Tychonoff} if $X$ is also Hausdorff)
  if for all $y \in X$ and every closed $Z \subset X \setminus \{y\}$ there
  exists $f : X \to [0,1]$ continuous such that $f(y) = 0$ and $f(z) = 1$ for
  all $z \in Z$.
  We say $y$ and $Z$ are separated by a (Urysohn) function.
\end{definition}

\begin{corollary}[Urysohn's Lemma]
  Every normal space is completely regular.
\end{corollary}

\begin{lemma}
  A compact (Hausdorff) space is normal hence completely regular.
\end{lemma}

\begin{proof}
  Fix disjoint closed $Y$ and $Z$ and let $y \in Y$.
  Consider the open cover of $Z$ given by $\{ V_{y,z} : z \in Z \}$
  where each $V_{y,z} \in N(z)$ is disjoint from some $U_{y,z} \in N(y)$
  (existence ensured by Hausdorff). By compactness,
  there exists a finite subcover $\{V_{y,z_i}\}_{i=1}^n$. For each of these
  $V_{y,z_i}$, let $U_{y, z_i} \in N(y)$ denote the corresponding disjoint
  neighborhood of $y$ and consider
  \[
    U'_y = \bigcap_{i=1}^n U_{y, z_i} \in N(y)
  \]
  $U'_y$ is open because it is the intersection of finitely many open sets.
  It is also disjoint from
  \[
    V'_y \coloneqq \bigcup_{i=1}^n V_{y, z_i}
  \]
  which contains $B$ and is also open.

  Now consider the open cover $\{U'_y : y \in Y\}$,
  let $\{U'_{y_i}\}_{i=1}^n$ be a finite subcover,
  and let $U = \cup_{i=1}^n U'_{y_i}$. Analogously,
  let $V = \cap_{i=1}^n V'_{y_i}$ where $V'_y$ is
  given above (open cover of $B$ and disjoint from $U'_y$).
  Then $U \supset Y$ and $V \supset Z$ provide two disjoint
  separating open sets.
\end{proof}

\begin{lemma}
  \label{lem:completely-regular-equals-initial-topo-cts}
  A topological space $(X, \tau)$ is completely regular (i.e.\ Tychonoff) space
  iff the original topology coincides with the initial topology $\sigma(X,
  \cC(X))$ i.e.\ the smallest topology that makes every function in $\cC(X)$
  continuous.
\end{lemma}

\begin{proof}
  We only show $\Rightarrow$. Let $U$ be $\tau$-open and for $x \in U$
  pick an Urysohn function $f \in \cC(X)$ such that $f(x) = 0$
  and $f(U^c) = 1$. Then $V_x = \{y : f(y) < 1\} = f^{-1}((-\infty, 1))$
  is a $\sigma(X, \cC(X))$-open neighborhood of $x$ contained
  in $U$, so $U = \cup_{x \in U} V_x$ is $\sigma(X, \cC(X))$-open.
  Since $\sigma(X, \cC(X))$ is minimal, we have $\tau = \sigma(X, \cC(X))$.
\end{proof}

\subsection{Radon, Borel, and Baire measures}

\begin{definition}
  A non-negative set function $m : 2^X \to [0, +\infty]$ on $X$
  is an \emph{outer measure on $X$} (or Carath\'eodory outer measure)
  if:
  \begin{enumerate}
    \item $m(\emptyset) = 0$
    \item $A \subset B \implies m(A) \leq m(B)$ (monotone)
    \item $m\left( \cup_{n=1}^\infty A_n\right) \leq \sum_{n=1}^\infty m(A_n)$
      for all $A_n \subset X$. (countable subadditivity)
  \end{enumerate}
\end{definition}

\begin{definition}
  Let $m : 2^X \to [0, +\infty]$ be a non-negative set function
  satisfying $m(\emptyset) = 0$.
  A set $A \subset X$ is \emph{Carath\'eodory measurable wrt
  $m$} (Carath\'eodory $m$-measurable) if for any $E \subset X$
  \[
    m(E) = m(E \cap A) + m(E \setminus A)
  \]
  We use $\fM_m$ to denote the class of all Carath\'eodory $m$-measurable sets.
\end{definition}

\begin{theorem}[Carth\'eodory construction]
  \begin{enumerate}
    \item $\fM_m$ is an algebra, $m$ is additive on $\fM_m$
    \item (Finite additivity) For all sequences of pairwise disjoint
      $A_i \in \fM_m$ and any $E \subset X$
      \begin{align*}
        m\left(E \cap \bigcup_{i=1}^n A_i\right) &= \sum_{i=1}^n m(E \cap A_i) \\
        m\left(E \cap \bigcup_{i=1}^\infty A_i\right)
        &= \sum_{i=1}^\infty m(E \cap A_i) + \lim_{n \to \infty} m\left(E \cap \bigcup_{i=n}^\infty A_i\right)
      \end{align*}
    \item If $m$ is an outer measure on $X$, then $\fM_m$ is a $\sigma$-algebra,
      $m$ is countably additive on $\fM_m$, and
      $m$ is complete on $\fM_m$
  \end{enumerate}
\end{theorem}

\begin{remark}
  The outer measure is constructed such that it satisfies
  countable additivity on the measurable sets $\fM_m$.
\end{remark}

\begin{example}
  Let $\fX$ be a family of subsets of $X$ such that $\emptyset \in \fX$.
  Given $\tau : \fX \to [0,+\infty]$ with $\tau(\emptyset) = 0$, set
  \[
    m(A) = \inf\left\{
      \sum_{n=1}^\infty \tau(X_n) : X_n \in \fX, A \subset \cup_{n=1}^\infty X_n
    \right\}
  \]
  where $\fM(A) = \infty$ in the absence of such sets $X_n$.
  Then $m$ is an outer measure, denoted $\tau^*$.

  This is where the ``outer'' comes from: $\cup_n X_n \supset A$
  is an outer approximation to $A$ using (potentially overlapping)
  sets from $\fX$ hence $\sum_{n=1}^\infty \tau(X_n)$ is an overapproximation
  to the ``size'' of $A$. $m(A)$ is the best (i.e. smallest) overapproximation.
\end{example}

Recall the \emph{Borel $\sigma$-algebra}, denoted $\cB(X)$, is generated by all
open sets.
\begin{definition}
  The \emph{Baire $\sigma$-algebra}, denoted by $\Ba(X)$,
  is generated by sets of the form
  \begin{align}
    \label{eq:functionally-open}
    \{x \in X : f(x) > 0 \}
  \end{align}
  where $f \in \cC(X)$ (called \emph{functionally open sets}).
\end{definition}

\begin{remark}
  $\Ba(X)$ is the smallest $\sigma$-algebra where every $f \in \cC(X)$ is
  measurable. It coincides (via a truncation and monotonicity argument) to
  the smallest one making every $f \in \cC_b(X)$ measurable.
  Contrast this to \cref{lem:completely-regular-equals-initial-topo-cts}, which
  shows that completely regular spaces are those with the smallest topology
  where every $f \in \cC(X)$ is continuous.
\end{remark}

\begin{remark}
  Since the functionally open sets can be written as $f^{-1}((0, \infty))$
  for continuous $f$, they are also Borel sets. Therefore,
  the class of Baire sets are contained in the class of Borel sets.
\end{remark}

\begin{lemma}
  \label{lem:metric-space-closed-set-variety}
  In a metric space $(X, d)$, any closed set $S$ is the set of zeros of a continuous
  function (namely $d_S(x) = \inf_{s \in S} d(x,s)$).
  Hence, $\cB(X) = \Ba(X)$.
\end{lemma}

\begin{lemma}[Baire sets are countably determined]
  Every $A \in \Ba(X)$ is determined by some countable family of functions,
  i.e. has the form
  \[
  A = \{x : (f_i(x))_{i=1}^\infty \in B \}
  \quad\text{for some}~f_i \in \cC(X), B \in \cB(\bR^{\aleph_0})
  \]
  Moreover, every set of this form is Baire and
  we can take $f_i \in \cC_b(X)$.
\end{lemma}

\begin{proof}
  We first show every set of the same form as $A$ is Baire.
  True if $B$ is closed, since \cref{lem:metric-space-closed-set-variety}
  allows us to write $B = \phi^{-1}(0)$ for some continuous function $\phi :
  \bR^{\aleph_0} \to \bR$ so $\psi = x \mapsto \phi((f_n(x))_{n \geq 1})$ is
  continuous hence $A = \psi^{-1}(0)$ is also closed. But this is the converse.

  For any fixed $\{f_n\}_{n \geq 1}$, the class of
  sets $B \in \cB(\bR^{\aleph_0})$ satisfying
  \[
    \{ x : (f_i(x))_{i \geq 1} \in B \} \in \Ba(X)
  \]
  is a $\sigma$-algebra containing
  $B = \prod_{i} (-\infty, a_i)$
  where $a_i \neq \infty$ for only finitely many $i$.
  This is a basis for $\cB(\bR^{\aleph_0})$, thus $\Ba(X)$
  contains it and the two coincide (recall $\Ba \subset \cB$ since
  functionally determined sets are $\cB$-open).

  On the other hand???
\end{proof}

A consequence of the monotone class theorem ???

Throughout, we consider (signed) measures of \emph{bounded variation}
unless explicitly denoted otherwise.

\begin{definition}
  Let $X$ be a topological space.
  \begin{itemize}
    \item A countably additive measure on $\cB(X)$ is called a
      \emph{Borel measure}
    \item A countably additive measure on $\cB a(X)$ is called a
      \emph{Baire measure}
    \item A Borel measure $\mu$ on $X$ is called \emph{Radon measure}
      if every $B \in \cB(X)$ can be approximated from the inside
      by compact sets: for $\eps > 0$ exists $K_\eps \subset B$
      such that $\lvert \mu \rvert(B \setminus K_\eps) < \eps$.
  \end{itemize}
\end{definition}

When are two Borel measures equal?
\begin{lemma}
  If two Borel measures coincide on all open sets, thne they
  coincide on all Borel sets.
\end{lemma}

\begin{proof}
  Split $\mu = \mu^+ - \mu^-$ and notice that each of the two
  components are nonnegative and coincide on open sets.
  By monotone class theorem, $\mu^+ = \mu^-$. \todo{finish}
\end{proof}

$\mu$ is Radon iff $\lvert \mu \rvert$ is Radon iff both $\mu^+$
and $\mu^-$ are Radon.

Inner and outer approximatino of measures on $\bR^n$:

\begin{theorem}
  $\mu \geq 0$ on $\cB(\bR^n)$, then any Borel set $B \subset \bR^n$ and
  any $\eps > 0$ exists $U_\eps$ open and $F_\eps$ closed such that
  $F_\eps \subset A \subset U_\eps$ and $\mu(U_\eps \setminus F_\eps) < \eps$.
\end{theorem}

\begin{proof}
  Let $\cA$ the class of all sets $A in \cB$ such that
  $F_\eps \subset A \subset U_e$ and $\mu(U_\eps \setminus F_\eps) < \eps$
  for some closed set $F_\eps$ and open set $U_\eps$.

  Every closed $A$ is in $\cA$, since we can take
  $F_\eps = A$ and $U_\eps$ some open $\delta$-nbd
  and consider $\delta \to 0$.

  It suffices to show that $\cA$ is a $\sigma$-algebra, since the closed
  sets generate $\cB$. $\cA$ is closed wrt complements, so it remains
  to verify closure under countable union.

  Let $A_j \in \cA$, $\eps > 0$. Then exists closed $F_j$
  and open $U-j$ such that $F_j \subset A_j \subset U-j$ and
  $\mu(U_j \setminus F_j) < \eps 2^{-j}$ for $j \in \bN$.

  The set $U = U_{j=1}^\infty U_j$ is open, and $Z_k = U_{j=1}^k F_j$ is closed.

  Observe $Z_k \subset \cup_{j=1}^\infty A_j \subset U$ and for
  sufficiently large $k$ $\mu(U \setminus Z_k) < \eps$.

  Indeed, $\mu(\cup_j^\infty \mu_j \setminus F_j) < 2^{}$ \todo{??}
\end{proof}

\begin{definition}
  Set function $\mu \geq 0$ defined on $\cA \subset 2^X$
  is \emph{tight} on $\cA$ if $\forall \eps > 0$ exists compact
  $K_\eps \subset X$ such that $\mu(A) < \eps$ for all $A \in \cA$
  that does not meet $K_\eps$.

  Additive set function $\mu$ of bounded variation on an algebra is
  \emph{tight} if its total variation $\lvert \mu \rvert$ is tight.
\end{definition}

A Borel measure is tight iff $\forall \eps > 0$ exists compact
$K_\eps$ such that $\lvert \mu \rvert(X \setminus K_\eps) < \eps$
(the ``total variation measure'').

The second definition is necessary to handle Baire sets.

\begin{definition}
  $\mu$ is \emph{regular} if $\forall A \in \cA, \eps > 0$, $\exists F_\eps$
  closed such that $F_\eps \subset A$, $A \setminus F_\eps \in \cA$,
  and \todo{??}
\end{definition}

Theorem 27 implies any Borel measure on $\bR_n$ is regular, and the
same proof works to show any Borel measure on metric space is regular.

\begin{corollary}[Baire measures are regular]
  \label{corr:baire-measure-regular}
  Every Baire measure $\mu$ on topological space $X$ is regular.
  Moreover, for every Baire set $E$ and $\eps > 0$,
  there exists a continuous function $f$ on $X$ such that
  $f^{-1}(0) \subset E$ and $\lvert \mu \rvert(E \setminus f^{_1}(0)) < \eps$.
\end{corollary}
