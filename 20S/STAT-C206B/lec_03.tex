\lecture{3}{2020-01-28}{Daniell integration}

\begin{theorem}[Extension to Radon measure]
  \label{thm:extend-tight-to-radon}
  Suppose an algebra $\cA$ of subsets of topological space $X$
  contains a base of the topology.
  Let $\mu$ be a regular additive set function of bounded variation on $\cA$.
  If $\mu$ is tight, then it admits a unique extension to a Radon measure on
  $X$.
\end{theorem}

\begin{proof}
  V.I. Bogachev, ``Measure Theory'' Theorem 7.3.2
\end{proof}


\begin{corollary}[Tight Baire measures extend to Radon]
  \label{corr:tight-baire-extend-radon}
  Let $X$ be a completely regular spaace.
  Then every tight Baire measure $\mu$ on $X$
  admits a unique extensino to a Radon measure.
\end{corollary}

\begin{proof}
  Every Baire measure is regular by \cref{corr:baire-measure-regular}.

  Since $X$ is completely regular,
  by \cref{lem:completely-regular-equals-initial-topo-cts}
  its topology is coincides with $\tau(X, \cC(X))$: the smallest making every
  function in $\cC(X)$ continuous.
  The functionally open sets form a base of this topology (they are the
  pullback of the base of open intervals for $\cB$ under all continuous
  functions $\cC(X)$), so
  \cref{thm:extend-tight-to-radon} yields the desired extension.
\end{proof}

% This allows us to extend measures on the Baire $\sigma$-field to measures
% on the Borel $\sigma$-field.

\begin{definition}
  A \emph{vector lattice of functions} is a linear space of
  real functions on a nonempty set $\Omega$ such that
  $\max(f,g) \in \cF$ for all $f,g \in \cF$.
\end{definition}

\begin{remark}
  Notice $\min(f,g) = \max(-f,-g) \in \cF$ and $\lvert f \rvert \in \cF$.
  Also, since $\max(f,g) = (\lvert f - g \rvert + f + g) / 2$
  it suffices to require $\cF$ be closed under absolute values.
\end{remark}

\begin{theorem}[Daniell integration]
  \label{thm:daniell-integration}
  Let $\cF$ be a vector lattice of functions on a set $\Omega$
  such that $\ind \in \cF$.
  Let $L$ be a linear functional on $\cF$ with:
  \begin{itemize}
    \item $L(f) \geq 0$ for all $f \geq 0$ (positive)
    \item $L(\ind) = 1$
    \item $L(f_n) \to 0$ for every $f_n \downarrow 0$
  \end{itemize}
  Then there exists a unique probability measure $\mu$ on
  $\cA = \sigma(\cF)$ generated by $\cF$ such that
  $\cF \subset \cL^1(\mu)$ and
  \[
    L(f) = \int_\Omega f d\mu,\qquad \forall f \in \cF
  \]
\end{theorem}

\begin{note}{Compare this with Riesz representation theorem}
  For $X$ a compact space, $L$ linear functional on $\cC(X)$
  with $L(\ind) = 1$ and $L(f) \geq 0$ for $f \geq 0$
  (positive linear functional),
  then $L(f) = \int_X f d\mu$
  with unique regular Borel probability measure $\mu$ on $X$.

  The relation is through Dini's theorem: If $\{f_n\} \subset \cC(X)$,
  $X$ compact, and $f_n(x) \downarrow 0$,
  then $\lim_{n \to \infty} \sup_{x \in X} f_n(x) = 0$.
\end{note}

\begin{proof}
  Denote $\cL^+$ the set of all bounded functions $f$
  of the form $f(x) = \lim_{n \to \infty} f_n(x)$, where
  $f_n \in \cF$ are nonegative and the sequence $\{f_n\}$ is increasing.
  $\{f_n\}$ is uniformly bounded, hence $\{L(f_n)\}$
  is increasing and bounded by properties of $L$ so by monotone convergence
  $\lim_n L(f_n)(x)$ exists for all $x$ and we can extend $L$
  to $f \in \cL^+$ by defining $L(f) = \lim_n L(f_n)$.

  We show that the extended functional $L(f)$ is well-defined, coincides
  on bounded nonnegative functions in $\cF$ with the original functional,
  and possesses the following properties:
  \begin{enumerate}
    \item $L(f) \leq L(g)$ for all $f,g \in \cL^+$ with $f \leq g$ (positive)
    \item $L(f + g) = L(f) + L(g)$, $L(cf) = c L(f)$ for all
      $f, g \in \cL^+$ and $c \in [0, +\infty)$ (linear)
    \item $\min(f,g) \in \cL^+$, $\max(f,g) \in \cL^+$, and
      \[
	L(f) + L(g) = L(\min(f,g)) + L(\max(f,g))
      \]
      for all $f, g \in \cL^+$
    \item $\lim_n f_n \in \cL^+$ for every uniformly bounded
      increasing sequence of functions $f_n \in \cL^+$,
      and $L(\lim_n f_n) = \lim_n L(f_n)$.
  \end{enumerate}

  Suppose $\{f_n\}$ and $\{g_k\}$ are two increasing sequences of nonnegative
  functions in $\cF$ with $\lim_n f_n \leq \lim_k g_k$.
  Then $\min(f_n, g_k) \in \cF$ are increasing to $f_n$
  (because $ f_n \leq \lim_n f_n \leq \lim_k g_k$) hence
  \[
    L(f_n) = \lim_k L(\min(f_n, g_k)) \leq \lim_k L(g_k)
  \]
  where the first equality follows from properties of $L$
  (take difference between successive $k+1$ and $k$ terms, use linearity
  and positivity and decreasing residual term) and the second
  because $g_k - \min(f_n, g_k) \geq 0$ for all $k$ and $L$ is positive
  and linear. Take $n \to \infty$ to conclude
  $\lim_n L(f_n) \leq \lim_k L(g_k)$.

  If $\{f_n\}$ and $\{g_k\}$ both converge to the same $f \in \cL^+$,
  then apply the above result symmetrically to get
  $\lim_n L(f_n) = \lim_k L(g_k)$, hence $L$ is well-defined
  on $\cL^+$. By considering constant sequences for $f \in \cF$,
  we have that $L$ coincides with the original on $\cF \cap \cL^+$
  and therefore properties (1) and (2) continue to hold by linearity.

  Property (3) is because for $f_n \uparrow f$ and $g_n \uparrow g$,
  $\cF \ni \min(f_n, g_n) \uparrow \min(f,g) \in \cL^+$
  (analogously for $\max$)
  and property (2) applied to
  \[
    f+g = \min(f,g) + \max(f,g)
  \]

  To verify (4), suppose $\cF \ni f_{m,n} \uparrow f_m \in \cL^+$
  (note the sequence $\{f_m\}$ is not in $\cF$, but each term is a limit
  of a sequence $\{f_{m,n}\}_n$ in $\cF$).
  Let $g_m = \max_{n \leq m} f_{m,n} \in \cF$, so
  $g_m \leq g_{m+1}$ and $f_{m,n} \leq g_m \leq f_m$ for $n \leq m$.
  Taking $n,m \to \infty$ shows $\lim_m f_m = \lim_m g_m \in \cL^+$
  so by well-definedness
  \[
    \lim_m L(f_m) = \lim_m L(g_m) = L(\lim_m g_m)
  \]
  But since $g_m$ and $f_k$ are both increasing, $\lim_k f_k - g_m \downarrow
  0$ so in fact (by property of $L$)
  \[
    \lim_m L(f_m) = L(\lim_m g_m) = L(\lim_k f_k)
  \]

  \begin{figure}[H]
    \centering
    \incfig{sketch-of-proof-of-4}
    \caption{Sketch of the inequalities involved in proving property (4)}
    \label{fig:sketch-of-proof-of-4}
  \end{figure}

  Armed with this extension of $L$ to $\cL^+$, we now define $\mu$.
  Denote by $\cG$ the class of all sets $G \subset \Omega$ with
  $\ind_{G} \in \cL^+$,
  and for $G \in \cG$ define $\mu(G) = L(\ind_{G})$.
  Notice that $\ind_{G \cap H} = \min(\ind_G, \ind_H) \in \cL^+$
  and $\ind_{G \cup H} = \max(\ind_G, \ind_H) \in \cL^+$
  by property (3), so $\cG$ is closed wrt finite unions and intersections.
  By property (4), it is also closed under countable unions.

  Furthermore, $\mu$ is a nonnegative monotone additive function on $\cG$,
  with inclusion-exclusion, i.e.
  \[
    \mu(G \cup H) - \mu(G \cap H) = \mu(G) + \mu(H)
  \]
  continuity from below, i.e. for $G_n \uparrow G$
  \[
    \mu(G_n) \uparrow \mu(G)
  \]
  and satisfies $\mu(\Omega) = 1$.

  Following \myref{eg:munroe-outer-meas} and closure of $\cG$
  under countable union, use $\mu$ to construct a (Munroe) outer measure
  \[
    \mu^*(A) = \inf\{\mu(G) : G \in \cG, A \subset G\}
  \]
  By \myref{thm:caratheodory-construction},
  $\mu^*$ is a countably additive measure on the $\sigma$-algebra
  \[
    \cB = \{B \subset \Omega: \mu^*(B) + \mu^*(\Omega \setminus B) = 1\}
  \]
  Let $\mu$ denote the restriction of $\mu^*$ to $\cB$.
  \begin{note}{Uncertain about above theorem}
    Should check details of section 1.5 Borgachev
  \end{note}

  Armed with $\mu$, we now verify that $\cA = \sigma(\cF)$ (the
  $\sigma$-algebra generated by our vector lattice of functions $\cF$)
  is contained in the domain $\cB$ where $\mu$ is defined.
  For $f \in \cL^+$, $\{f > c\} \in \cG$ for all $c$ because
  \begin{align}
    \label{eq:superlevel-set-in-G}
    \ind_{\{f > c\}} = \lim_n \min(1, n \max(f - c, 0))
  \end{align}
  Hence $f \in \cL^+$ are measurable wrt $\sigma(\cG)$, but they are also
  measurable wrt $\sigma(\cF)$ (since they are monotone limits of things in
  $\cF$), so $\cG \subset \sigma(\cL^+) = \sigma(\cF)$ and by Dynkin
  $\pi$-$\lambda$ we have $\sigma(\cG) = \sigma(\cF) = \cA$. Thus, it suffices
  to show $\cG \subset \cB$.

  For $G \in \cG$, let $\cF \ni f_n \uparrow \ind_{G}$ so
  \[
    \mu^*(G) = \mu(G) = \lim_{n \to \infty} L(f_n)
  \]
  and since (because $\mu^*$ is an outer measure)
  $\mu^*(G) + \mu^*(\Omega \setminus G) \geq 1$, to show
  $G \in \cB$ it suffices to prove $\mu^*(G) + \mu^*(\Omega \setminus G) \leq 1$
  i.e.
  \[
    \mu^*(\Omega \setminus G) \leq \lim_n L(\ind - f_n)
  \]
  Let $U_c = \{\ind - f_n > c\}$ for $n \in \bN$ and $c \in (0,1)$,
  so $U_c \supset \Omega \setminus G$
  (by monotonicity we must have $(\ind - f_n)(x) \equiv 1$ for $x \not\in G$)
  and $\ind_{U_c} &\leq c^{-1}(\ind - f_n)$ by definition.
  We also have $U_c \in \cG$ by \cref{eq:superlevel-set-in-G},
  so altogether
  \begin{align*}
    \mu^*(\Omega \setminus G) \leq \mu(U_c) &= L(\ind_{U_c}) \leq c^{-1} L(1 - f_n)
  \end{align*}
  Take $c \to 1$ and $n \to \infty$ to get the desired result.

  Having defined $\mu$ on $\cA = \sigma(\cF)$, it remains to prove
  $\cF \subset \cL^1(\mu)$ and that $L(f) = \int_\Omega f d\mu$.
  For $f \in \cL^+$ with $f \leq 1$ (bounded), approximate $f$
  as the limit of increasing sequence of simple functions
  \begin{align*}
    f_n &= \sum_{j=1}^{2^n - 1} j 2^{-n} \ind\{j 2^{-n} < f < (j+1) 2^{-n}\} \\
    L(f_n) &= \sum_{j=1}^{2^n - 1} j 2^{-n} \mu\{j 2^{-n} < f < (j+1) 2^{-n}\}
    = \int_\Omega f_n d\mu
  \end{align*}
  By property (4) and properties of the integral on 
  increasing sequences $\{f_n\}$, taking $n \to \infty$ yields
  the desired formula $L(f) = \int_\Omega f d\mu$.
  By considering truncations $\cL^+ \ni \min(f,n) \to f$,
  which are increasing in $n$, this extends to non-negative $f$.
  By splitting $f \in \cF$ as $f = \max(f,0) - \max(-f, 0)$, this
  shows that $\cF \subset \cL^1(\mu)$ with the desired integration formula.

  Lastly, the uniqueness of $\mu$ follows from Dynkin's $\pi$-$\lambda$
  combined with the fact that it is uniquely determined on the class $\cG$,
  which is closed wrt finite intersections and generates $\cA$ as a
  $\sigma$-algebra.
\end{proof}


