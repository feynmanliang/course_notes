\lecture{5}{2020-02-04}{Extreme point representation of measures}

Throughout this section, let $E$ be a topological vector space
over $\bR$.

\begin{definition}
  For convex $A \subset E$, an \emph{open segment} is
  a subset of type
  \[
    \{(1-\lambda) a + \lambda b : \lambda \in (0,1) \},\qquad a\neq b \in A
  \]
  ``Open'' means that $\lambda \not\in \{0,1\}$.

  $x_0 \in A$ is an \emph{extreme point} of $A$ if it belongs to no open segment
  of $A$, i.e.
  \begin{align*}
    (\exists \lambda \in [0,1] : x_0 = (1-\lambda) a + \lambda b)
    \implies x_0 = a \text{ or } x_0 = b
  \end{align*}
  A closed subet $B \subset A$ is an \emph{extreme subset} of $A$ if
  for $a,b \in A$
  \begin{align*}
    (\exists \lambda \in (0,1) : (1-\lambda) a + \lambda b \in B)
    \implies \{a,b\} \subset B
  \end{align*}
\end{definition}

\begin{figure}[ht]
    \centering
    \incfig[0.7]{extreme-points}
    \caption{On a triangle, the extreme points are the vertices and the extreme
    subsets are the sides}
    \label{fig:extreme-points}
\end{figure}

In finite dimensions, it is clear that optimizing a linear objective function
(i.e.\ sliding a hyperplane along its normal vector until it gets to the edge
of $A$, see \cref{fig:extreme-points}) will always yield an extreme set. The
following lemma makes this explicit for more general settings.

\begin{lemma}
  \label{lem:cvx-subset-has-extreme-subset}
  Let $E$ be a real Hausdorff TVS, $A \subset E$ nonempty compact convex,
  $f \in E'$ a continuous linear functional on $E$, $\beta = \inf_{x \in A} f(x)$.
  Then $B = A \cap f^{-1}(\beta)$ is nonempty, compact, extreme subset of $A$.
\end{lemma}

\begin{proof}
  Since $A$ is compact and $f$ is continuous, $f(A)$ is compact
  in $\bR$ and $\beta \in f(A)$, so $\beta = f(a)$ for some $a \in A$
  and thus $B$ is nonempty.

  Since $f$ is linear continuous and $\{\beta\}$ is convex closed,
  $f^{-1}(\{\beta\})$ is convex closed and hence
  $B$ is also convex closed. As a closed subset of compact set $A$,
  $B$ is also compact.

  To show $B$ is extreme, suppose $(1-\lambda)a + \lambda b \in B$
  for $a,b \in A$ and $\lambda \in (0,1)$.
  Suppose towards contradiction $a \not\in B$.
  Then $f(a) > \beta$ by definition of $B$ and so by linearity
  \[
    f((1-\lambda) a + \lambda b)
    = (1 - \lambda) f(a) + \lambda f(b) > (1 - \lambda) \beta + \lambda \beta
    = \beta
  \]
  contradicting $(1-\lambda)a + \lambda b \in B$.
\end{proof}

The following theorem is our goal for today.
It says that compact convex sets are equal to the closed convex hulls of their
extreme points.
In $\bR^d$, the analogous result is Carath\'eodory's theorem which says
that every $x \in \conv(P) \subset \bR^d$ for a set $P$ can be written
as a convex combination of $d+1$ points in $P$.

\begin{theorem}[Krien-Milman]
  \label{thm:krien-milman}
  Let $E$ be a real Hausdorff locally convex topological vector space.
  Then every nonempty compact convex subset $A \subset E$
  is the closed convex hull in $E$ of the set of extreme points of $A$.
\end{theorem}

\begin{proof}
  We first show each nonempty extreme subset $X \subset A$ contains an extreme
  point of $A$. Let $\fX$ consist of nonempty extreme subsets of $A$
  contained in $X$.
  $\fX$ is nonempty (by \cref{lem:cvx-subset-has-extreme-subset}),
  so partially order $\fX$ by inclusion.
  The intersection of any descending chain in $\fX$ is a nonempty set because
  it is the intersection of nonempty compact (extreme sets are closed by
  definition, closed subsets of compact sets are compact) sets (c.f. Cantor's
  intersection theorem). It is also still extreme hence in $\fX$,
  hence by Zorn's lemma $\fX$ possesses a minimal element, say $Y$.

  To see that $Y$ is a singleton, suppose otherwise. Then $Y$ would contain
  $x$ and $y$ distinct and since $E$ is locally convex, by the
  Hahn-Banach separation theorem there exists $f \in E'$ such that $f(x) <
  f(y)$.
  By \cref{lem:cvx-subset-has-extreme-subset}, $Z = Y \cap f^{-1}(\inf f(Y))$
  is a nonempty extreme subset of $A$ that does not contain $y$,
  contradicting minimality of $Y$.

  Next, let $B$ be the closed convex hull in $E$ of the set of all extreme
  points of $A$.
  We will show $A = B$ by showing $A \setminus B$ is empty.
  Suppose towards contradiction $x_0 \in A \setminus B$.
  Since $B$ is convex non-empty, by the Hahn-Banach separation theorem there
  exists $f \in E'$ such that $\inf_B f(x) > f(x_0)$.
  Then by \cref{lem:cvx-subset-has-extreme-subset},
  $W = A \cap f^{-1}(\inf_A f(x))$ is a nonempty extreme subset of $A$
  and is disjoint from $B$ since
  \[
    \inf_A f(x) \leq f(x_0) < \inf_B f(x)
  \]
  However, by the previous part $W$ would contain an extreme point of $A$,
  which is a contradiction since $W \cap B = \emptyset$.
\end{proof}

Whereas \myref{thm:krien-milman} states that every convex set can be
represented by its extreme points, the following proposition addresses
the case where the set $K$ is compact but not necessarily convex.

\begin{proposition}
  \label{prop:cvx-hull-contains-extreme-pts}
  Suppose $E$ is Hausdorff real LCTVS, $K \subset E$ compact whose closed
  convex hull $A = \overline{\conv(K)}$ is also compact.
  Then each extreme point of $A$ belongs to $K$.
\end{proposition}

\begin{figure}[ht]
    \centering
    \incfig{cvx-hull-contains-extreme-pts}
    \caption{Sketch of the proof of \cref{prop:cvx-hull-contains-extreme-pts}}
    \label{fig:cvx-hull-contains-extreme-pts}
\end{figure}

\begin{proof}
  Let $x \in A$ be an extreme point.
  For any closed convex neighborhood $U$ of $0 \in E$,
  by compactness (which implies total boundedness) of $K$,
  for some $\{a_i\}_1^n \subset K$
  \[
    K \subset \cup_{i=1}^n (a_i + U)
  \]
  Let $A_i = \overline{\conv(K \cap (a_i + U))}$.
  Each $A_i$ is compact (because $A_i \subset K$ is closed and $K$ is compact),
  so $\overline{\conv(\cup_i^n A_i)}$ is compact and since
  $K \subset \overline{\conv(\cup_i^n A_i)} \subset A = \overline{\conv(K)}$
  by minimality of convex hull we must have $A = \overline{\conv(\cup_i^n A_i)}$.

  Hence, $x = \sum_i^n \lambda_i x_i$ is a convex combination of $x_i \in A_i$.
  As $x \in A$ is extreme point, $x = x_i$ for some $i$.
  Thus, $x \in A_i \subset a_i + U$,
  so $x \in K + U$.
  Since $K$ is closed and $U$ is an arbitrary neighborhood of $0$,
  $x \in K$ as desired.
\end{proof}

\begin{example}
  The set of extreme points of a compact convex set $A$
  need not be closed (hence why we take closed convex hull),
  even when $E$ is finite dimensional!

  Let $E = \bR^3$ and
  \[
    S = \{ (x,y,z) : z = 0, (x-1)^2 + y^2 = 1\}\qquad
    \text{and }
    A
    = S \cup \{(0,0,1), (0,0,-1)\}
  \]
  Then $0$ is not an extreme point, but is the limit of
  many sequences of extreme points.

  \begin{figure}[H]
    \centering
    \incfig{extreme-points-not-closed}
    \caption{$0$ is not an extreme point, but every point on
      $S \setminus \{0\}$ is. Hence, the set of extreme points
      does not contain a limit point and is not closed.
    }
  \end{figure}
\end{example}

\begin{example}
  A compact convex set $A$ need not be the convex hull of its extreme points.
  Take $E = \ell^\infty$, $e_n = \delta_n \in E$, $A$ the closed
  convex hull in $E$ of $e_n / n$ for $n \in \bN$.
  By \cref{prop:cvx-hull-contains-extreme-pts}, the extreme points
  of $A$ are $0$ and the points $\{e_n / n\}_{n \in \bN}$.
  $A$ is compact and contains all points $x = \sum_{n \geq 1} \lambda_n e_n / n$,
  $\lambda_n$ a convex combinations.

  TODO: finish
\end{example}

\begin{definition}
  For a Banach space $(X, \|\cdot\|_X)$, let $X'$ be the set of bounded
  linear functionals $L : X \to \bR$, i.e.
  $\sup_{x \in X} \lvert L(x) \rvert / \|x\| \leq c$ for some $c \geq 0$
  $\iff$ $L$ is linear and continuous.

  Given $L \in X'$, write $\|L\|_{X'}$ for smallest $c$
  that works $\|L\|_{X'} = \inf \{\lvert l(x) \rvert : \|x\| = 1\}$.

  FACT: $(X', \|\cdot\|_{X'})$ is a Banach space.
  Each $x \in X$ defines a linear map $X' \to \bR$ via evaluation
  \[
    e_x = L \mapsto L(x)
  \]
  The \emph{weak-$\ast$ topology} on $X'$ is the weakest/coarsest
  (i.e. initial topology) $\tau(X', \{e_x\}_{x \in X})$ that makes all of these
  maps continuous.
\end{definition}

\begin{example}[Riesz representation theorem]
  Let $S$ be a compact Hausdorff space,
  $X = \cC(S)$ continuous functions from $S$ to $\bR$.
  $X$ is a Banach space with the $\sup$-norm $\|x\| = \sup_{s \in S} \lvert x(s) \rvert$.
  Then $X'$ finite signed memasures on $S$.
  associated with any finite signed measure $\mu$.
  is the continuous linear functional $L(x) = \int_S x(s) \mu(ds)$.

  We want to know what is $\|\cdot\|_{X'} = \|\cdot\|_{\mu(S)}$?
  If $L(x) = \int_S x(s) \mu(ds)$ then by definition
  \begin{align*}
    \|L\|_{X'}
  &= \sup\{ \lvert L(x) \rvert : \|x\|_X = 1 \} \\
  &= \sup\left\{ \left\lvert \int_S x(s) \mu(ds) \right\rvert : \sup_{s \in S} \lvert x(s) \rvert = 1 \right\} \\
  &= \sup\left\{ \left\lvert \int_S x(s) \mu^+(ds)  - \int_S x(s) \mu^-(ds) \right\rvert : -1 \leq \lvert x(s) \rvert \leq 1 \right\}
\end{align*}
We know $\mu^+$ and $\mu^-$ are perpendicular; their supports are disjoint, so
\begin{align*}
  \|L\|_{X'}
  &= \sup\left\{ \left\lvert
      \int_{S^+} x(s) \mu^+(ds)  - \int_{S^-} x(s) \mu^-(ds)
  \right\rvert : -1 \leq \lvert x(s) \rvert \leq 1 \right\}
\end{align*}
Take $x = \ind_{S^+} - \ind_{S^-}$ to conclude
$\|L\|_{X'} = \mu^+(S^+) + \mu^-(S^-) = \mu^+(S) + \mu^-(S) = \lvert \mu \rvert(S) = \|\mu\|_{TV}$
Hence, $(\cC(S), \|\cdot\|_{\infty})$ has dual
$(M(S), \|\cdot\|_{TV})$ and the weak-$\ast$ topology on $M(S)$
is the weakest/coarsest/smallest topology that makes continuous
all maps $\mu \mapsto \int_S x(s) \mu(ds)$ for $x \in \cC(S)$.
Notice that this is strictly weaker than the TV topology, because for example
two unit point masses have TV norm 2 but
$\int x(s) \delta_{s'}(ds) = x(s') \approx x(s'') = \int x(s) \delta_{s''}(ds)$
when $s' \approx s''$.
\end{example}

This is the story for compact spaces. What about for only locally compact spaces?

Let $T$ Hausdorff LC, $M(T)$ real bounded signed Radon measures on $T$.
\begin{definition}
  $\cC_0(T)$ are the continuous functions vanishing at infinity, i.e.
  $f \in cC(T)$ such that $\lim_{x \to \infty} f(\pm x) = 0$.
\end{definition}

View $M(T)$ as the (Banach) dual of $C_0(T)$ equipped with weak-$\ast$
topology.
The set $M_+^1(T)$ of positive measures in $M(T)$ having total mass at most one
is compact and convex, because:

\begin{theorem}[Banach-Alaoglu]
  The unit ball in $(X', \|\cdot\|_{X'})$ is compact
  in the weak-$\ast$ topology.
\end{theorem}

We will show that the extreme points of $M_+^1(T)$ of are $0$ and the
Dirac measures $\delta_t$ for $t \in T$.

It is clear that $0$ is an extreme point of $M_+^1(T)$.

Suppose $\mu \neq 0$ is another element in $M_+^1((T)$, then it suffices
show $K = \supp \mu$ is a single point. If $t_1 \neq t_2 \in K$, then
by Hausdorffness choose $U_1 \ni t_1$ and $U_2 \ni t_2$ disjoint.
Then $m = \mu(U_1)$ satisfies $0 < m < 1$.

Define two measures $\alpha = (\mu \vert U_1) / m$ ($\mu$ restricted and renormalized)
and $\beta = (\mu - m \alpha) / (1 - m)$ what's left over after subtracting $m \alpha$.
Then $\alpha, \beta \in M_+^1(T)$, $\alpha \neq \beta$, and $\mu = m \alpha + (1 - m) \beta$,
contradicting $\mu$ extremal. Hence, $K$ is a single point.

A similar argument shows that if $T$ is compact, then the set of positive
measures of unit total mass is compact and convex, and that its extreme points
are the Dirac measures $\delta_t$.

\begin{theorem}[Stone-Weierstrass]
  Let $E$ be a subalgebra of $C(S)$. Suppose
  $E$ separates points and contains constants, then $E$ is dense in $C(S)$.
\end{theorem}


