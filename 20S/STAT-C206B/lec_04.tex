\lecture{4}{2020-01-30}{Representation theorems}

Recall our current setup:
\begin{itemize}
  \item $\cF$ is a vector lattice of functions over $\Omega$
    containing constants, i.e.
    $\ind \in \cF$
  \item $\cL^+$ consists of $f$ such that $0 \leq f_n \uparrow f < \infty$
    for $f_n \in \cF$
  \item $\cG = \{G \subset \Omega : \ind_G \in \cL^+\}$ are the subsets
    whose indicators can be realized as monotone limits within $\cF$
    (i.e. can be well approximated using $\cF$, which we can use $L$ to
    measure); we use $\cG$ as the approximating set when constructing
    the Munroe outer measure
  \item For $G \in \cG$, define $\mu(G) = L(\ind_G) = \lim_n L(f_n)$
    and extend using \myref{eg:munroe-outer-meas} and
    \myref{thm:caratheodory-construction}
    to the class $\fM_{\mu^*} = \cB$ which contains
    $\cA = \sigma(\cF) = \sigma(\cG)$.
\end{itemize}

\begin{corollary}
  Suppose that in \cref{thm:daniell-integration} the vector lattice $\cF$ is
  closed wrt uniform convergence.
  Let $\cG_\cF$ be the class of sets of the form
  $\{f > 0\}$, $f \geq 0$, $f \in \cF$.
  Then $\cG_\cF$ generates $\cA = \sigma(\cF)$ and we have
  \begin{align*}
    \mu(A) &= \inf\{\mu(G) : A \subset G, G \in \cG_\cF\}, &&\forall A \in \cA \\
    \mu(G) &= \sup\{L(f) : f \in \cF, 0 \leq f \leq \ind_G\}, &&\forall G \in \cG_\cF
  \end{align*}
\end{corollary}


\begin{proof}
  Suffices to verify $\cG_\cF$ equals the $\cG$ introduced during the
  theorem's proof, since we showed $\sigma(\cG) = \cA$.
  Taking $c=0$ in \cref{eq:superlevel-set-in-G} shows
  $\{f > 0\} \in \cG$ for all non-negative $f \in \cF$.
  On the other hand, if $G \in \cG$ then $f_n \uparrow \ind_G$
  for some $f_n \geq 0$ in $\cF$. Letting $f = \sum_{n=1}^\infty 2^{-n} f_n$,
  by uniform convergence of the series we have $f \in \cF$.
  Clearly $f \geq 0$ and $G = \{f > 0\}$.
\end{proof}

A general fact about vector lattices where signed measures decompose into
a positive part and negative part. If $\nu$ is a signed measure on $\Omega$,
then $\nu = \nu_+ - \nu_-$ for $\nu_{\pm}$ unique nonnegative meaasures with
disjoint supports. Its total variation decomposes as:
\[
  \|\nu\| = \nu_+(\Omega) + \nu_-(\Omega)
\]


\begin{theorem}
  Let $\cF$ be a vector lattice of bounded functions on a set $\Omega$
  such that $\ind \in \cF$.
  Suppose that we are given a linear functional $L$ on $\cF$ that is continuous
  wrt $\|f\| = \sup_{\Omega} \lvert f(x) \rvert$, i.e.
  \[
    \|L\| = \inf \{ c : \|L(f)\| \leq c \|f\|~\forall f \in \cF \} < \infty
  \]
  Then $L$ can be represented as $L = L^+ - L^-$ where
  $L^+ \geq 0$, $L^- \geq 0$, and for all nonnegative $f \in \cF$ we have
  \begin{align*}
    L^+(f) = \sup_{0 \leq g \leq f} L(g),\qquad
    L^-(f) = -\inf_{0 \leq g \leq f} L(g)
  \end{align*}
  In addition, letting $\lvert L \rvert = L^+ + L^-$, we have for $f \geq 0$
  \begin{align*}
    \lvert L \rvert(f) = \sup_{0 \leq \lvert g \rvert \leq f} \lvert L (g) \rvert,
    \qquad \|L\| = L^+(1) + L^-(1)
  \end{align*}
\end{theorem}

\begin{proof}
  Given two nonnegative $f,g \in \cF$ and $h \in \cF$ such that
  $0 \leq h \leq f + g$, can write $h = h_1 + h_2$ where
  $0 \leq h_1 \leq f$, $0 \leq h_2 \leq g$, $h_1, h_2 \in \cF$.
  Just let $h_1 = \min(f,g)$ and $h_2 = h - h_1$ and verify.

  Let $L^+$ be defined by the previous theorem. We first
  show additivity on nonnegative functions.
  For $f,g \in \cF$ nonnegative, we have
  \begin{align*}
    L^+(f+g) = \sup \{L(h) : 0 \leq h \leq f + g\}
    = \sup\{ L(h_1) + L(h_2) : 0 \leq h_1 \leq f, 0 \leq h_2 \leq g\}
    = L^+(f) + L^+(g)
  \end{align*}
  where we used the previous decomposition.

  Now we show additivity on arbitrary functions.
  Let $f = f_1 - f_2$, where $f_1, f_2$ non-negative. There might be
  multiple decompositions for the samae $f$, but still
  \[
    L^+(f) = L^+(f_1) - L^+(f_2)
  \]
  since $f_1 + f^- = f_2 + f^+$ and we showed $L^+$ is additive on nonnegative
  functions.

  Define $L^- =  L^+ - L$ and since $L^+(f) \geq L(f)$ for $f \geq 0$
  we have that $L^-$ is also nonnegative.

  Finally,
  \begin{align*}
    \|L\| \leq \|L^+\| + \|L^-\| \\
    &= L^+(1) + L^-(1) \\
    &= 2L^+(1) - L^-(1) \\
    &= \sup\{L(2\psi-1) : 0 \leq \psi \leq 1 \} \\
    &\leq \sup \{ L(h) : -1 \leq h \leq 1 \} \\
    &\leq \|L\|
  \end{align*}
\end{proof}

\begin{corollary}
  \label{corr:meas-repr-decreasing-clf}
  Suppose in addition $L(f_n) \to 0$ for every
  $f_n \downarrow 0$. Then $L^+$ and $L^-$
  share this property as well, and are defined by nonnegative
  countably additive measures on $\sigma(\cF)$
  and $L$ has representation
  \[
    L(f) = \int_\Omega f d\mu, \qquad \forall f \in \cF
  \]
  with some signed countably additive measure $\mu$ on $\sigma(\cF)$.
\end{corollary}

\begin{proof}
  TODO
\end{proof}

Here is an analogue of the Riesz representation theorem:

\begin{theorem}
  \label{thm:baire-meas-repr-clf}
  Let $X$ be a topological space. The formula
  \[
    L(f) = \int_X f d\mu
  \]
  establishes a one-to-one correspondence between Baire measures $\mu$
  on $X$ and continuous linear functionals $L$ on $\cC_b(X)$ with
  the property
  \[
    \lim_n L(f_n) = 0
  \]
  for every $f_n \downarrow f$.
\end{theorem}

\begin{proof}
  Any measure $\mu$ on $\Ba(X)$ defines a continuous linear functional
  on $\cC_b(X)$ through the above formula.

  Converse follows from \cref{corr:meas-repr-decreasing-clf}.
\end{proof}

See ``Banach limit''

\begin{theorem}[Dini's theorem]
  \label{thm:dini}
  On a compact space $K$,
  if $\{f_n\} \subset \cC(X)$ converges pointwise decreasing to zero,
  then $\{f_n\}$ converges in the Banach space $\cC(X)$ to $0$, i.e.
  converges uniformly to zero.
\end{theorem}

We get a Riesz representation for compact spaces:

\begin{theorem}[Riesz representation theorem]
  On a compact Hausdorff space $K$,
  every continuous linear functional $L$
  on the Banach space $C(K)$ has a unique
  Radon measure $\mu$ such that
  \begin{align*}
    L(f) = \int_K f d\mu, \qquad \forall f \in \cC(K)
  \end{align*}
\end{theorem}

\begin{proof}
  By \cref{thm:dini}, TODO
\end{proof}

From now, we assume $S$ to be locally compact, second countable, and Hausdorff
(lcscH). Let $\cG, \cF, \cK$ denote open, closed, and compact sets in
$S$
and put $\hat{\cG} = \{ G \in \cG, \bar{G} \in \cK\}$.
Let $\hat{C}_+ = \hat{C}_+(S)$ denote the class of continuous functions
$f : S \to \bR_+$ with compact support (i.e.
closure of the set $\{x \in S; f(x) > 0\}$).

We want to extend the idea of invariant (Haar) measure from just groups
to more general spaces such as the sphere.

\begin{theorem}[Riesz representation]
  \label{thm:riesz-extend-fts-to-meas}
  If $S$ is lcscH, then every positive linear functional $\mu$
  on $\hat{C}_+(S)$ extends uniquely to a measure on $S$ that assigns
  finite mass to compact sets.
\end{theorem}

\begin{proof}
  Kallenberg, ``Foundations of modern probability''
\end{proof}

\begin{theorem}
  On every lcscH group $G$ there exists, uniquely up to normalization,
  a left-invariant measure $\lambda \neq 0$ that assigns finite mass to
  compact sets.
  If $G$ is compact, then $\lambda$ is also right-invariant.
\end{theorem}

\begin{proof}
  Kallenberg, ``Foundations of modern probability''
\end{proof}

\begin{definition}
  Given group $G$ and space $S$, a \emph{left action}
  of $G$ on $S$ is a mapping $(g,s) \mapsto gs$ such that $es = s$
  and $(gh)s = g (hs)$ for any $g,h \in G$ and $s \in S$,
  where $e$ denotes the identity element in $G$.

  Similarly, a \emph{right action} is a mapping $(s,g) \mapsto s g$
  satisfying similar compatibility conditions.

  The action is \emph{transitive} if for all $s,t \in S$ there
  exists $g \in G$ such that $g s = t$ or $s g = t$ respectively.

  All actions are assumed left henceforth.
\end{definition}

When $G$ is a topological group, we assume the action is
a continuous $G \times S \to S$ map.

\begin{definition}
  $h : G \to S$ is \emph{proper} if $h^{-1} K$ is
  compact in $G$ for any compact $K \subset S$.

  If this holds for all $\pi_s(x) = x s$, $s \in S$,
  we say the group action is proper.
\end{definition}

\begin{definition}
  A memasure $\mu$ on $S$ is $G$-invariant if
  $\mu(xB) = \mu B$ for any $x \in G$
  and $B \in \cS$. This is clearly equivalent to
  \[
    \int f(xs) \mu(ds) = \mu f
  \]
  for any measurable $f : S \to \bR_+$ and $x \in G$.
\end{definition}

\begin{theorem}
  If we have lcshH group $G$ acting transively and properly on lcscH space $S$.
  Then there exists a unique (up to normalization) $G$-invariant
  measure $\mu \neq 0$ on $S$ which assigns finite mass to compact sets.
\end{theorem}

\begin{proof}
  We first show existence.
  Fix $p \in S$ and let $\pi = x \mapsto x p : G \to S$.
  Letting $\lambda$ be a left Haar measure on $G$, define
  the pushforward $\mu = \lambda \circ \pi^{-1}$ on $S$.
  Since $\pi$ is proper and the Haar measure on $G$ assigns finite mass
  to compact sets, $\mu$ is a measure on $S$ that assigns finite mass to
  compact sets. To see $G$-invariance, for $f \in \hat{C}_+(S)$ and
  $x \in G$
  \[
    \int_S f(xs) \mu(ds)
    = \int_G f(xyp) \lambda(dy)
    = \int_G f(yp) \lambda(dy)
    = \mu f
  \]
  by invariance of $\lambda$.

  Now we consider uniqueness. Let $\mu$ be abitrary $G$-invariant
  measure on $S$ assigning finite mass to compact sets. Define the subgroup
  \[
    K = \{x \in G : x p = p \} = \pi^{-1}\{p\}
  \]
  (the stabilizer of $p$, subgroup leaving $p$ fixed) and note $K$ is
  compact (since $\pi$ is proper). Let $\nu$ be the normalized Haar
  measure on $K$, and define
  \[
    \bar{f}(x) = \int_K f(xk) \nu(dk), \quad x \in G, f \in \hat{C}_+(G)
  \]
  At each point $x$, $\bar{f}$ takes $f$
  and ``smooths things out'' using $K$ translated to $x$.

  If $xp = yp$ then $y^{-1} x p = p$ and so $y^{-1} x \eqqcolon h \in K$
  which implies $x = y h$. Hence, left invariance of $\nu$ yields
  \[
    \bar{f}(x) = \bar{f}(yh)
    = \int_K f(yhk) \nu(dk)
    = \int_K f(yk) \nu(dk)
    = \bar{f}(y)
  \]
  So the mapping $f \mapsto f^*$ given by
  \[
    f^*(s) = \bar{f}(\pi^{_1}\{s\}) \equiv \bar{f}(x),\quad s = xp \in S, x \in G, f \in \hat{C}_+(G)
  \]
  is well defined, and for any $B \subset (0, \infty)$ we have
  \[
    (f^*)^{-1} B = \pi (\bar{f}^{-1} B) \subset \pi[(\supp f) \cdot K]
  \]
  where $(\supp f) \cdot K$ is the support of $f$ ``convolved with $K$''
  via the group action. Hence, the RHS is compact (both $\supp f$ and $K$ compact)
  and since $\pi$ and the action are continuous.
  Therefore $f^*$ has compact support.

  Also, $\bar{f}$ is continuous (by group operation cts and
  dominated convergence), so $\bar{f}^{-1}[t,\infty)$
  is closed and hence compact for every $t > 0$.

  ??? So $f^*$ is something we can integrate against $\mu$.

  We may now define functional $\lambda$ on $\hat{C}_+(G)$
  by $\lambda f = \mu f^*$ for $f \in \hat{C}_+(G)$.
  Linearity and positivity of $\lambda$ are clear from the corresponding
  properties of the mapping $f \mapsto f^*$ and the measure $\mu$.
  We note that $\lambda$ is finite on $\hat{C}_+(G)$ since $\mu$ is locally
  finite, so by \cref{thm:riesz-extend-fts-to-meas} we can extend $\lambda$ to a measure
  on $G$ that assigns fintie mass to compact sets.

  To see $\lambda$ left invariant, for $f \in \hat{C}_+(G)$ and define
  $f_y(x) = f(yx)$. THen for $s = xp \in S$ and $y \in G$ we have
  \[
    f^*_y(s) = \bar{f}_y(x)
    = \int_K \bar{f}_y(xk) \nu(dk)
    = \bar{f}(yx)
    = f^*(ys)
  \]
  Hence by invariance of $\mu$ we have
  \[
    \int_G f(yx) \lambda(dx) = \lambda f_y = \mu f_y^* = \int_S f^*(ys) \mu(ds) = \mu f^* = \lambda f
  \]
  So $\lambda$ is the Haar measure.

  Now fix $g \in \hat{C}_+(S)$ and put $f(x) = g(xp) = g \circ \pi(x)$ for $x \in G$.
  THen $f \in \hat{C}_+(G)$ because $\{f > 0\} \subset \pi^{-1} \supp g$
  which is compact since $\pi$ is proper. By definitino of $K$,
  for $s = xp \in S$ we have
  \[
    f^*(s) = \bar{f}(x) = \int_K f(xk) \nu(dk)
    = \int_K g(xkp) \nu(dk)
    = \int_K g(xp) \nu(dk)
    = g(s)
  \]
  so we've found an inverse for the $*$ operation, so
  \[
    \mu g = \mu f^* = \lambda f = \lambda(g \circ \pi) = (\lambda \circ \pi^{-1}) g
  \]
  which shows $\mu = \lambda \circ \pi^{-1}$. Since $\lambda$ is unique up to
  normalization, the same thing is true for $\mu$.
\end{proof}
