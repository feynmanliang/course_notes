\lecture{9}{2020-02-18}{}

Notation from Kallenberg ``Foundations of Modern Probability.''

Underlying probability space $(\Omega, A, P)$.

$E^\cF[\cdot] = E[\cdot \mid \cF]$ conditional probability given
$\cF$ a sub-$\sigma$-field of $\cA$


\begin{definition}
  The \emph{conditional probability} of an event $A \in \cA$ given
    $\sigma$-field $\cF$ is
    \[
      P^{\cF}A = E^{\cF} \ind_A
    \]
    Equivalently
    \[
      P[A \mid \cF] = E[\ind_A \mid \cF],\quad A \in \cA
    \]
\end{definition}

Thus $P^\cF A$ is the almost surely unique random variable in $L^1(\cF)$ satisfying
\[
  E[P^\cF A; B] = P(A \cap B) \quad \forall B \in \cF
\]
$E[X \mid \cF]$ is the a.s. unique $\cF$-measurable RV $Y$
such that $E[X \ind_B] = E[Y \ind_B]$ for all $B \in \cF$.

\begin{itemize}
  \item $P^\cF A = P A$ a.s. iff $A \perp \cF$ (denoting independent)
  \item $P^\cF A = \ind_A$ a.s. if $A$ agrees a.s. with a set in $\cF$, i.e.
    $P(A \Delta B) = 0$ for $B \in \cF$
  \item Positivity of $E^\cF$ implies $0 \leq P^\cF A \leq 1$ a.s.
  \item Monotone convergence property gives
    \[
      P^\cF \bigcup_n A_n = \sum_n P^\cF A_n
    \]
    a.s., for $A_i \in \cA$ disjoint
\end{itemize}

We would like to have an assignment $A \mapsto (P^\cF A)(\omega)$
that for each fixed $\omega$ is a probability measure
(c.f. regular conditional distributions).

\begin{definition}
  A \emph{kernel} between two measurable spaces $(T, \cT)$
  and $(S, \cS)$ is a function $\mu : T \times S \to \overline{\bR_+}$
  such that $\mu(t, B)$ is $\cT$-measurable in $t \in T$ for fixed $B \in \cS$
  and a measure in $B \in \cS$ for fixed $t \in T$.

  $\mu$ is a \emph{probability kernel} if $\mu(t, S) = 1$ for all $t$

  Kernels on the basic probability space $\Omega$ are called
  \emph{random measures}.
\end{definition}

For fixed $B \in \cS$, $t \mapsto \mu(t, B)$ as a function
$(T, \cT) \to (\overline{\bR_+}, \cB)$ is measurable.
For fixed $t \in T$, the set function $B \mapsto \mu(t, B)$
is a measure on $(S, \cS)$.

A random measure on $(S, \cS)$ is a map $\nu : \Omega \times \cS \to \overline{\bR_+}$
such that for each $B \in \cS$, $\omega \mapsto \nu(\omega, B)$ is $\cA$-measurable
(i.e. $\nu(\cdot, B)$ is a random variable) and for each $\omega \in \Omega$,
$B \mapsto \nu(\omega, B)$ is a measure on $(S, \cS)$ (i.e. $\nu(\omega, \cdot)$
is a measure).

Fix a $\sigma$-field $\cF \subset \cA$ and random element $\xi$ in some measurable
space $(S, \cS)$.
\begin{definition}
  A \emph{regular conditional distribution} of $\xi$, given $\cF$,
  we mean a version of the function $P[\xi \in \cdot \mid \cF]$ on
  $\Omega \times \cS$ which is a probability kernel from $(\Omega, \cF)$
  to $(S, \cS)$, hence an $\cF$-measurable random probability measure
  on $S$.
\end{definition}

If $\eta \in (T, \cT)$, we can talk about the rcd of $\xi$ given $\eta$
as a random measure of the form
\[
  \mu(\eta, B) = P[\xi \in B \mid \eta]~\text{a.s.},\quad Bb \in \cS
\]
where $\mu$ is a probability kernel $T \to S$.

In the extreme cases $\xi$ is $\cF$-measurable or independent of $\cF$,
$P[\xi \in B \mid \cF]$ has the regular version $\ind\{\xi \in B\}$
or $P\{\xi \in B\}$ respectively.

\begin{definition}
  $(S, \cS)$ is a \emph{Borel space} if there exists
  a Borel subset $B \subset \bR$ such that if we equip $B$ with the trace
  $\sigma$-field
  \[
    \cB(B) = \{ B \cap C : C \in \cB(\bR)\} = \{ A \in \cB(\bR) : A \subset B \}
  \]
  then there is a bijection $f : S \to B$ that is measurable with a measurable
  inverse.
\end{definition}

\begin{example}
  Any complete seperable metric space (with its Borel $\sigma$-field)
  is a Borel space. For example, $\bN^\infty$.
\end{example}

\begin{theorem}
  For any Borel space $S$ and measurable space $T$, let $\xi \in S$
  and $\eta \in T$ be random elements. Then there exists a unique
  probability kernel $\mu$ from $T$ to $S$ satisfying
  $P[\xi \in \cdot \mid \eta] = \mu(\eta, \cdot)$ a.s.,
  and $\mu$ is unique a.e. $\cL(\eta)$.
\end{theorem}

\begin{proof}
  By assumption of Borelness, wlog assume $S \in \cB(\bR)$.
  For every $r \in \bQ$ we may choose some measurable
  $f_r = f(\cdot, r) : T \to [0,1]$ such that
  \[
    f(\eta, r) = P[\eta \leq r \mid \eta]~\text{a.s.}
  \]
  \todo{Finish with Lebesgue-Stieltjes measures}
\end{proof}

\begin{theorem}[Disintegration]
  Fix measurable spaces $S$ and $T$, $\sigma$-field $\cF \subset \cA$,
  random $\xi \in S$ such that $P[\xi \in \cdot \mid \cF]$ has a regular
  version $\nu$.
  Further consider $\cF$-measurable random $\eta \in T$ and
  measurable $f : S \times T \to \overline{\bR}$
  with $\bE \lvert f(\xi, \eta) \rvert < \infty$. Then
  \[
    \bE[f(\xi, \eta) \mid \cF] \aseq \int \nu(ds) f(s, \eta)
  \]
\end{theorem}

\begin{proof}
  In the case when $\cF = \sigma(\eta)$ and
  $P[\xi \in \cdot \mid \eta] = \mu(\eta, \cdot)$
  for some probability kernel $\mu$
  from $T$ to $S$, then this becomes
  \[
    E[f(\xi, \eta) \mid \eta] \aseq \int \mu(\eta, ds) f(s, \eta)
  \]
  Integrating, we get the commonly used formula
  \[
    E f(\xi, \eta) = E\int\nu(ds) f(s, \eta) = E \int \mu(\eta, ds) f(s, \eta)
  \]
  If $\xi \perp \eta$, we can take $\mu(\eta, \cdot) \equiv \cL(\xi)$
  deterministic and the above reduces to the relation to previous lemma.


  To prove the theorem, let $B \in \cS$ and $C \in \cT$. Use averaging property of conditional
  expectations to get
  \todo{?????}
\end{proof}

\begin{definition}[Conditional independence]
  $\cF_i, \ldots, \cF_n, \cG \subset \cA$ sub-$\sigma$-fields are
  \emph{conditionally indepenent} given $\cG$ if
  \[
    P^\cG \bigcap_{k \leq n} B_k \aseq \prod_{k \leq n} P^\cG B_k
  \]
  wehre $B_k \in \cF_k$.

  For infinite collections $\{\cF_t\}_{t \in T}$, require the same
  property for every finite subcollection
  $\{\cF_{t_i}\}_{i=1}^n$ with distinct indices.

  Use $\perp_\cG$ to denote pairwise conditional independence
  given $\cG$.
\end{definition}

\begin{proposition}[Conditional independence, Doob]
  For $\cF, \cG, \cH$ $\sigma$-fields, $\cF \perp_{\cG} \cH$
  iff
  \[
    P[H \mid \cF, \cG] \aseq P[H \mid \cG]
  \]
  for all $H \in \cH$.
\end{proposition}

\begin{proof}
  Assuming above and using the chain and pull-out properties of
  conditional expectations, we get for $F \in \cF$, $H \in \cH$
  \begin{align*}
    P^\cG(F \cap H)
    &= E^\cG P^{\cF \lor \cG}(F \cap H) = E^\cG[ P^{\cF \lor \cG} H; F] \\
    &= E^\cG[ P^\cG H; F] = (P^\cG F)(P^\cG H)
  \end{align*}
  where $\cF \lor \cG$ is the smallest $\sigma$-field containing
  both $\cF$ and $\cG$. This shows $\cF \perp_\cG \cH$.

  Conversely, assume $\cF \perp_\cG \cH$ and using the chain and
  pull-out properties, we get for any $F \in \cF$, $G \in \cG$,
  and $H \in \cH$
  \[
    E[P^\cG H; F \cap G]
    = E[(P^\cG F)(P^\cG H); G]
    = E[P^\cG(F \cap H); G]
    = P(F \cap G \cap H)
  \]
  By a monotone class argument, this extends to
  \[
    E[P^\cG H; A] = P(H \cap A), \qquad A \in \cF \lor \cG
  \]
  and the result follows by the averaging characterizataion of
  $P^{\cF \lor \cG} H$.
\end{proof}

From this result, we can conclude some further useful properties.
Let $\bar{\cG}$ denote the completion of $\cG$ wrt
$\cA$, generated by $\cG$ and $\cN = \{ N \subset A \}$ \todo{??}.

\begin{corollary}
  For any $\sigma$-fields $\cF$, $\cG$, and $\cH$, we have
  \begin{itemize}
    \item $\cF \perp_{\cG} \cH$ iff $F \perp_\cG (\cG, \cH)$
    \item $\cF \perp_\cG \cF$ iff $\cF \subset \overline{\cG}$
  \end{itemize}
\end{corollary}

\begin{proof}
  By proposition ??, both relations are equivalent to
  \[
    P[F \mid \cG, \cH] \aseq P[F \mid \cG],\quad F \in \cF
  \]
\end{proof}

\begin{proposition}[Chain rule]
  For any $\cG$, $\cH$, and $\cF_i$, $i \in [n]$, TFAE
  \begin{itemize}
    \item $H \perp_{\cG} (\cF_i)_{i}$
    \item $\cH \perp_{\cG, \cF_1, \ldots, \cF_n} \cF_{n+1}$ for all $n \geq 0$
  \end{itemize}
\end{proposition}

In particular, we have the commonly used equivalence
\[
  \cH \perp_{\cG} (\cF, \cF') \iff \cH \perp_{\cG} \cF, \cH \perp_{\cG, \cF} \cF'
\]

\begin{definition}
  An \emph{extension} of $(\Omega, \cA, P)$ is a product space
  $(\hat{\Omega}, \hat{\cA}) = (\Omega \times S, \cA \otimes \cS)$
  equipped with a probability measure $\hat{P}$ satisfying
  $\hat{P}(\cdot \times S) = P$.
\end{definition}

Any random element $\xi \in \Omega$ can be regarded as a function on
$\hat\Omega$, so we can replace $\xi$ with $\hat\xi(\omega,s) = \xi(\omega)$
which clearly has the same distribution.

For extensions of this type, we can retain our original notation
and write $P$ and $\xi$ instead of $\hat{P}$ and $\hat\xi$.

\begin{lemma}[Extension]
  Fix probability kernel $\mu$ between measurable $S$ and $T$.
  let $\xi \in S$ be random element.
  Then there exists random $\eta \in T$ defined on some extension of
  $\Omega$ such that
  $P[\eta \in \cdot \mid \xi] \aseq \mu(\xi, \cdot)$
  and also $\eta \perp_\xi \zeta$ for all $\zeta$ on $\Omega$.
\end{lemma}

\begin{proof}
  Let $(\hat\Omega, \hat\cA) = (\Omega \times T, \cA \otimes \cT)$.
  Define $\hat{P}$ by
  \[
    \hat{P} A
    = \int_\Omega P(d\omega) \int_T \ind_A(\omega, t) \mu(\xi(\omega), dt)
    = E \int \ind_A(\cdot, t) \mu(\xi, dt), \qquad A \in \hat\cA
  \]
  Clearly $\hat{P}(\cdot \times T) = P$ and $\eta(\omega, t) \equiv t$
  on $\hat\Omega$ satisfies
  $\hat{P}[\eta \in \cdot \mid \cA] \aseq \mu(\xi, \cdot)$.
  In particular, $\eta \perp_\xi \cA$ by
  Proposition ?? hence $\eta \perp_\xi \zeta$.
\end{proof}

\begin{lemma}
  $\mu$ probability kernel $S$ to Borel space $T$.
  Then exists measurable $f : S \times [0,1] \to T$ such that
  if $\theta \sim U(0,1)$ then
  $f(s,\theta)$ has distribution $\mu(s,\cdot)$ for every $s \in S$.
\end{lemma}

Will use this to prove the transfer principle:

\begin{theorem}
  $\xi \deq \tilde{\xi}$, $\eta$ random elements in $S$ and $T$ resp.
  THen exists $\tilde{\eta} \in T$ with
  $(\tilde\xi, \tilde{\eta}) \deq (\xi,\eta)$.
\end{theorem}
