% Some basic packages
%\usepackage[utf8]{inputenc}
\usepackage[english]{babel}
\usepackage[T1]{fontenc}

% TODO
\usepackage{stmaryrd}
\SetSymbolFont{stmry}{bold}{U}{stmry}{m}{n}

\usepackage[margin=1in]{geometry}
\usepackage{textcomp}
\usepackage{url}
\usepackage{graphicx}
\usepackage{float}
\usepackage{booktabs}
\usepackage{enumitem}


% Don't indent paragraphs, leave some space between them
\usepackage{parskip}

% Hide page number when page is empty
\usepackage{emptypage}

\usepackage{subcaption}
\usepackage{multicol}
\usepackage{xcolor}

% Math stuff
\usepackage{amsmath, amsfonts, mathtools, amsthm, amssymb}

% \mathscr fansy script capitals
\usepackage{mathrsfs}
% bbm fonts \mathbbm
\usepackage{bbm}

\usepackage{braket}
\usepackage{cancel}
\usepackage[bold]{hhtensor}

% bold shortcuts
\newcommand{\bC}{\ensuremath{\mathbb{C}}}
\newcommand{\bE}{\ensuremath{\mathbb{E}}}
\newcommand{\bN}{\ensuremath{\mathbb{N}}}
\newcommand\bQ{\ensuremath{\mathbb{Q}}}
\newcommand{\bR}{\ensuremath{\mathbb{R}}}
\newcommand{\bT}{\ensuremath{\mathbb{T}}}
\newcommand{\bZ}{\ensuremath{\mathbb{Z}}}

% caligraphic shortcuts
\newcommand{\cA}{\mathcal{A}}
\newcommand{\cB}{\mathcal{B}}
\newcommand{\cC}{\mathcal{C}}
\newcommand{\cD}{\mathcal{D}}
\newcommand{\cE}{\mathcal{E}}
\newcommand{\cF}{\mathcal{F}}
\newcommand{\cG}{\mathcal{G}}
\newcommand{\cH}{\mathcal{H}}
\newcommand{\cI}{\mathcal{I}}
\newcommand{\cK}{\mathcal{K}}
\newcommand{\cL}{\mathcal{L}}
\newcommand{\cM}{\mathcal{M}}
\newcommand{\cN}{\mathcal{N}}
\newcommand{\cO}{\mathcal{O}}
\newcommand{\cP}{\mathcal{P}}
\newcommand{\cS}{\mathcal{S}}
\newcommand{\cT}{\mathcal{T}}
\newcommand{\cU}{\mathcal{U}}
\newcommand{\cV}{\mathcal{V}}
\newcommand{\cX}{\mathcal{X}}

% fraktur shortcuts
\newcommand{\fG}{\mathfrak{G}}
\newcommand{\fm}{\mathfrak{m}}
\newcommand{\fM}{\mathfrak{M}}
\newcommand{\fX}{\mathfrak{X}}

% vector and matrix shortcuts
\newcommand{\va}{\vec{a}}
\newcommand{\vb}{\vec{b}}
\newcommand{\vc}{\vec{c}}
\newcommand{\vf}{\vec{f}}
\newcommand{\vg}{\vec{g}}
\newcommand{\vk}{\vec{k}}
\newcommand{\vmu}{\vec{\mu}}
\newcommand{\vp}{\vec{p}}
\newcommand{\vq}{\vec{q}}
\newcommand{\vu}{\vec{u}}
\newcommand{\vv}{\vec{v}}
\newcommand{\vw}{\vec{w}}
\newcommand{\vx}{\vec{x}}
\newcommand{\vy}{\vec{y}}
\newcommand{\vz}{\vec{z}}
\newcommand{\valpha}{\vec{\alpha}}
\newcommand{\vbeta}{\vec{\beta}}
\newcommand{\vsigma}{\vec{\sigma}}
\newcommand{\vxi}{\vec{\xi}}
\newcommand{\mA}{\matr{A}}
\newcommand{\mB}{\matr{B}}
\newcommand{\mD}{\matr{D}}
\newcommand{\mI}{\matr{I}}
\newcommand{\mK}{\matr{K}}
\newcommand{\mL}{\matr{L}}
\newcommand{\mM}{\matr{M}}
\newcommand{\mP}{\matr{P}}
\newcommand{\mQ}{\matr{Q}}
\newcommand{\mS}{\matr{S}}
\newcommand{\mU}{\matr{U}}
\newcommand{\mV}{\matr{V}}
\newcommand{\mX}{\matr{X}}
\newcommand{\mY}{\matr{Y}}
\newcommand{\mZ}{\matr{Z}}
\newcommand{\mSigma}{\matr{\Sigma}}
\newcommand{\mLambda}{\matr{\Lambda}}

% misc shortcuts
\newcommand{\simiid}{\overset{\text{iid}}{\sim}}
\newcommand{\eps}{\varepsilon}
\newcommand{\ind}{\mathbbm{1}}
\newcommand{\pto}{\overset{p}{\to}}
\newcommand{\wto}{\overset{w}{\to}}
\newcommand{\asto}{\overset{as}{\to}}
\newcommand{\aseq}{\overset{as}{=}}
\newcommand{\deq}{\overset{d}{=}}
\newcommand{\Dir}{\mathrm{Dir}}
\newcommand{\Gam}{\mathrm{Gamma}}
\DeclareMathOperator{\Var}{var}
\DeclareMathOperator{\Cov}{Cov}
\DeclareMathOperator{\Corr}{Corr}
\DeclareMathOperator{\conv}{conv}
\DeclareMathOperator{\sgn}{sgn}
\DeclareMathOperator{\supp}{supp}
\DeclareMathOperator{\diag}{diag}

% Easily typeset systems of equations (French package)
\usepackage{systeme}

% Put x \to \infty below \lim
\let\svlim\lim\def\lim{\svlim\limits}

%Make implies and impliedby shorter
\let\implies\Rightarrow
\let\impliedby\Leftarrow
\let\iff\Leftrightarrow
\let\epsilon\varepsilon

% Add \contra symbol to denote contradiction
\usepackage{stmaryrd} % for \lightning
\newcommand\contra{\scalebox{1.5}{$\lightning$}}

% \let\phi\varphi

% Command for short corrections
% Usage: 1+1=\correct{3}{2}

\definecolor{correct}{HTML}{009900}
\newcommand\correct[2]{\ensuremath{\:}{\color{red}{#1}}\ensuremath{\to }{\color{correct}{#2}}\ensuremath{\:}}
\newcommand\green[1]{{\color{correct}{#1}}}

% horizontal rule
\newcommand\hr{
    \noindent\rule[0.5ex]{\linewidth}{0.5pt}
}

% hide parts
\newcommand\hide[1]{}

% si unitx
\usepackage{siunitx}
\sisetup{locale = FR}

% Environments
\makeatother


% hide remreset package warning, https://tex.stackexchange.com/questions/438543/what-to-do-when-an-actively-maintained-package-requires-an-obsolete-package
\RequirePackage{silence}
\WarningFilter{remreset}{The remreset package}

% load thmtools but fix "proof proof" bug with thmbox
\usepackage{letltxmacro}
\LetLtxMacro\amsproof\proof
\LetLtxMacro\amsendproof\endproof
\usepackage{thmtools}
\AtBeginDocument{%
  \LetLtxMacro\proof\amsproof
  \LetLtxMacro\endproof\amsendproof
}

\declaretheorem[thmbox=M]{theorem}
\declaretheorem[thmbox=M,sibling=theorem]{proposition}
\declaretheorem[thmbox=M,sibling=theorem]{lemma}
\declaretheorem[thmbox=M,sibling=theorem]{corollary}
\declaretheorem[thmbox=M,sibling=theorem]{conjecture}
\declaretheorem[
    thmbox={style=M,bodystyle=\normalfont},
    sibling=theorem,
]{definition}
\declaretheorem[
    thmbox={style=M,bodystyle=\normalfont},
    sibling=theorem,
]{example}
\declaretheorem[style=remark,sibling=theorem]{remark}
\declaretheorem[style=remark,sibling=theorem]{exercise}

% use \cutthm{off,on} to prevent page breaks within thmbox
\makeatletter
\newcommand{\cutthmoff}{\thmbox@cutfalse}
\newcommand{\cutthmon}{\thmbox@cuttrue}
\makeatother

% End env with a small diamond (just like proof environments end with a small square)
% \usepackage{etoolbox}
% \AtEndEnvironment{example}{\null\hfill$\diamond$}%

% Fix some spacing
% http://tex.stackexchange.com/questions/22119/how-can-i-change-the-spacing-before-theorems-with-amsthm
\makeatletter
\def\thm@space@setup{%
  \thm@preskip=\parskip \thm@postskip=0pt
}


%% Exercise
%% Usage:
%% \exercise{5}
%% \subexercise{1}
%% \subexercise{2}
%% \subexercise{3}
%% gives
%% exercise 5
%%   exercise 5.1
%%   exercise 5.2
%%   exercise 5.3
%\newcommand{\exercise}[1]{%
%    \def\@exercise{#1}%
%    \subsection*{exercise #1}
%}

%\newcommand{\subexercise}[1]{%
%    \subsubsection*{exercise \@exercise.#1}
%}


% \incfig{} to import from figures/*.pdf_latex
\usepackage{import}
\usepackage{pdfpages}
% \usepackage{transparent}
\newcommand{\incfig}[2][1]{%
    \def\svgwidth{#1\columnwidth}
    \import{./figures/}{#2.pdf_tex}
}

% \lecture starts a new lecture (les in dutch)
%
% Usage:
% \lecture{1}{di 12 feb 2019 16:00}{Inleiding}
%
% This adds a section heading with the number / title of the lecture and a
% margin paragraph with the date.

% I use \dateparts here to hide the year (2019). This way, I can easily parse
% the date of each lecture unambiguously while still having a human-friendly
% short format printed to the pdf.

\usepackage{xifthen}
\def\testdateparts#1{\dateparts#1\relax}
\def\dateparts#1 #2 #3 #4 #5\relax{
    \marginpar{\small\textsf{\mbox{#1 #2 #3 #5}}}
}

\def\@lecture{}%
\newcommand{\lecture}[3]{
    \ifthenelse{\isempty{#3}}{%
        \def\@lecture{Lecture #1}%
    }{%
        \def\@lecture{Lecture #1: #3}%
    }%
    \section{\@lecture}
    \marginpar{\small\textsf{\mbox{#2}}}
}



\usepackage{fancyhdr}
\pagestyle{fancy}

\usepackage{comment}
\usepackage{todonotes} % todonotes
\usepackage{tcolorbox} % inline notes in fancy boxes
\tcbuselibrary{breakable} % make boxes breakable
\newenvironment{note}[1]{\begin{tcolorbox}[
    arc=0mm,
    colback=white,
    colframe=white!60!black,
    title=#1,
    fonttitle=\sffamily,
    breakable
]}{\end{tcolorbox}}

% Fix some stuff
% %http://tex.stackexchange.com/questions/76273/multiple-pdfs-with-page-group-included-in-a-single-page-warning
% \pdfsuppresswarningpagegroup=1

\author{Feynman Liang\thanks{\texttt{feynman@berkeley.edu}} \\ Department of Statistics, UC Berkeley}
\date{Last updated: \today}


\usepackage{nameref,hyperref}
\usepackage[capitalize]{cleveref}

% use \myref{} to refer to theorem by name if available
\makeatletter
\newcommand{\myref}[1]{\cref{#1}\mynameref{#1}{\csname r@#1\endcsname}}
\newcommand{\Myref}[1]{\Cref{#1}\mynameref{#1}{\csname r@#1\endcsname}}
\def\mynameref#1#2{%
  \begingroup
    \edef\@mytxt{#2}%
    \edef\@mytst{\expandafter\@thirdoffive\@mytxt}%
    \ifx\@mytst\empty\else
    \space(\nameref{#1})\fi
  \endgroup
}
\makeatother

\usepackage{csquotes}
\usepackage[
backend=biber,
style=alphabetic,
natbib=true
]{biblatex}

\usepackage[nottoc]{tocbibind}
\usepackage[toc,page]{appendix}

% show labels for \ref in margins
\usepackage{seqsplit}
\usepackage{showkeys}
\usepackage{xstring}
\renewcommand*\showkeyslabelformat[1]{%
\noexpandarg%
% instead of \textvisiblespace you can also put in ~
% if you want to keep a plain space at space characters
\StrSubstitute{\(\{\)#1\(\}\)}{ }{\textvisiblespace}[\TEMP]%
\parbox[t]{\marginparwidth}{\raggedright\normalfont\small\ttfamily\expandafter\seqsplit\expandafter{\TEMP}}}
