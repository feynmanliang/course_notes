\lecture{1}{2020-01-21}{A high level overview}

\begin{description}
  \item[Instructor] Jim Pitman, Evans 303
  \item[Text] Durrett's PTE, Version Jan 2019,
    \url{https://services.math.duke.edu/~rtd/PTE/PTE5_011119.pdf}
  \item[Topics]~
    \begin{itemize}
      \item Mainly chapters 5--9 of Durrett's
      \item Material from chapter 3 neglected in 205A
      \item Poisson processes
      \item Infinitely divisible and stable laws
    \end{itemize}
    We will start with Markov chains; read Chapter 5 of Durrett in
    next few weeks.
\end{description}


\subsection{What is a stochastic process?}

\textbf{Answers}

\begin{itemize}
  \item Distribution with time
  \item Family of r.v.s indexed by some index set $I$, often
    $I = \text{time}, \text{space}, \text{space-time}$.
  \item Usually some structure on $I$
    \begin{itemize}
      \item Metric space
      \item $I = \{1,\ldots,n\} = [n]$ (Julia and MATLAB's choice)
        \begin{itemize}
          \item Sometimes $I = \{0, 1,\ldots,n\}$ (Python's choice), be careful
        \end{itemize}
      \item Semi-group: $i,j \in I \implies i+j \in I$
        \begin{itemize}
          \item Sometimes want a group, e.g. $I = \bZ$
        \end{itemize}
    \end{itemize}
  \item A random function $I \to (\text{some space})$
\end{itemize}

Let's unpack that last answer a bit more.
Here and elsewhere, let $(\Omega, \cF, \Pr)$ be a (background) probability space.
For fixed $\omega \in \Omega$, the \emph{sample path}
$(X_i(\omega), i \in I)$ is a deterministic function
from $I$ to some space. To define a random function
taking values in the sample paths, we must be ensure that
\[
  \Omega \ni \omega \mapsto (X_i(\omega), i \in I)
\]
is measurable.

\begin{figure}[ht]
  \centering
  \incfig[0.5]{1-21-1}
  \caption{One sample path $(B_t(\omega))_{t \in I}$}
  \label{fig:1-21-1}
\end{figure}

\begin{example}
  Let $I = [n]$ and suppose $(X_i, i \in I)$ are real-valued.
  Is
  $\Pr(X_i, i \in I \text{ is increasing}) = \Pr(X_1 \leq \cdots \leq X_n)$
  the $\Pr$ of a measurable set?

  Notice
  \begin{align*}
    \Pr(X_1 \leq \ldots \leq X_n)
    &= \Pr(X_{i+1} - X_i \geq 0, i \in [n-1]) \\
    &= \Pr\left(
      \bigcap_{i \in [n-1]}
      \underbrace{\{X_{i+1} - X_i \geq 0\}}_{\eqqcolon E_i}
    \right)
  \end{align*}
  Why is $E_i$ measurable?
  $E_i$ is the preimage of $[0, \infty)$ under
  $\omega \mapsto X_2(\omega) - X_1(\omega)$,
  so it suffices to show measurability of this function.
  But this is true because the subtraction function $(x,y) \mapsto y - x$ is
  measurable (because it is continuous) and composition of measurable functions
  remain measurable.

  \begin{figure}[H]
    \centering
    \incfig[0.8]{1-21-2}
    \caption{
      $\omega \mapsto X_2(\omega) - X_1(\omega)$
      can be written as a composition of two measurable functions
    }
    \label{fig:1-21-2}
  \end{figure}
\end{example}

For yet another viewpoint, consider the first half of the composition from the
previous example and notice that the pair of r.v.s $(X_1, X_2)$ induces a
product $\sigma$-algebra $\cF_1 \otimes \cF_2$ generated by sets of the form
\begin{align}
  F_1 \times F_2
    &= \{(x_1, x_2) : x_i \in F_i \}
\end{align}
for $F_i \in \cF_i$ i.e. $X_i^{-1}(F_i) \in \cF$.
This is a very convenient generating set since (due to closure under
intersection of $\sigma$-fields $\cF_i$) it is closed under intersection and
hence a $\pi$-system, enabling the use of Dynkin's $\pi$-$\lambda$ theorem
(taking the $\lambda$-system to be all subsets where two candidate
distributions agree) to conclude the uniqueness of the distribution on the
product space $\cF_1 \otimes \cF_2$. In this view, a stochastic process
can be viewed as the unique measure on the product space
$\prod_{i \in I} \supp X_i$.

In conclusion, we have seen that a stochastic process can be viewed as:
\begin{itemize}
  \item A large collection of r.v.'s $(X_i, i \in I)$
  \item (With suitable formalism) a random function
    $\omega \mapsto (X_i(\omega), i \in I)$ from $\Omega$
    to the product space $\prod_{i \in I} \supp X_i$
  \item The (unique) probability measure (aka \emph{law}) of the random
    function on the product space (or a suitable subset,
    e.g. continuous functions $\subset \{ f : \bR \to \bR \}$)
    \begin{itemize}
      \item This is an example of the \emph{push-forward} of $\Pr$
        under $\omega \mapsto (X_i(\omega), i \in I)$, which induces a
        probability measure on the product space
      \item This allows us to forget about the background
        probability space $(\Omega, \cF, \Pr)$. Instead, take $\Omega$ to be
        the product space, $\cF$ the product $\sigma$-fields, and $\Pr$ the law
        of the push-forward measure.
    \end{itemize}
\end{itemize}

\textbf{Important advanced idea} (French idea --- Meyer \& school late 1960s):
You should think of the bivariate function $(\omega, i) \mapsto X(\omega, i)$
as the stochastic process.

\subsection{Major classes of stochastic processes}

\begin{enumerate}
  \item IID model, $(X_i, i \in I)$ are IID;
    law is product of identical probability laws.
  \item Independent, not necessarily identical;
    law is product of various probability laws.  Notice there's no problem if
    the laws are defined with respect to different $\sigma$-fields.
  \item Sums of IID or independent non-identically distributed;
    random walks with independent increments $S_n = \sum_{i=1}^n X_i$
  \item Martingales; centering the $X_i$ in the previous example
    (i.e.\ $\bE X_i = 0$) makes $S_n$ a martingale.

    Recall that a filtration is an ascending chain of $\sigma$-algebras
    $\cF_n \uparrow \cF$. For absolutely integrable random variable $X$,
    the conditional expectation (with respect to $\cF_n$) is the
    (almost surely) unique random variable $X_n = \bE[ X \mid \cF_n ]$
    which integrates like $X$ over $\cF_n$ (important slogan), i.e.
    \begin{align*}
      \bE[Y_n X_n ] &= \bE [ Y_n X ]
    \end{align*}
    is true for all $Y_n$ that are $\cF_n$-measurable (and for which the above
    makes sense i.e.  remains integrable).

    $(M_n, \cF_n)$ is a martingale $\iff \bE \lvert M_n \rvert < \infty$,
    $M_n \in \cF_n$, and $\bE[M_{n+1} \mid \cF_n] = M_n$. Iterating, we
    have for all $i < j$ we have $\bE_{\cF_i} M_j = \bE[M_j \mid \cF_i] = M_i$

    The notation $\bE_{\cF_i}$ for conditional expectation
    is used to suggest viewing $\bE_{\cF_i}$ as an operator
    $L^1(\Omega, \cF, \Pr) \to L^1(\Omega, \cF, \Pr)$.
    In fact, $\bE_{\cG}$ is a partial averaging operator:
    \[
      \cG \subset \cH \implies \bE_\cG \bE_\cH = \bE_{\cG} = \bE_{\cH} \bE_\cG
    \]
    As a consequence, $\bE_{\cG}$ is a projection:
    $\bE_{\cG} \bE_{\cG} = \bE_{\cG}$.

    \begin{figure}[ht]
      \begin{center}
        \incfig[0.5]{1-21-3}
      \end{center}
      \caption{The conditional expectation $\phi(x) = \bE[Z(X,Y) \mid X=x]$ can
        be viewed as a function of $x$ partially averages over $Y$ to return
        $\phi(x) = \bE[Z(x, Y)]$}
      \label{fig:1-21-3}
    \end{figure}

    \item Markov Chains / processes; ``chain'' usually implies either
      the state space $S$ or time set $I$ is discrete, whereas
      process often implies at least one (often both) $I$ and $S$
      are continuous (often intervals).

      The key idea for definition of a Markov Process is conditional
      independence (wortth understanding in detail). Abstractly,
      discrete r.v.s $(X,Y,Z)$ taking values in $(S_X, S_Y, S_Z)$ respectively
      is a \emph{Markov triple} iff $X$ and $Z$ are conditionally independent
      given $Y$, written $X \overset{Y}{\perp} Z$ means:
      \begin{enumerate}
        \item $\Pr[X=x, Z=z \mid Y=y] = \Pr[X=x \mid Y=y] \Pr[Z=z \mid Y=y]$
          for all $x,y,z$ such that this makes sense
        \item $\Pr[Z=z \mid X=x, Y=y] = \Pr[Z=z \mid Y=y]$
        \item $\Pr[X=x \mid Z=z, Y=y] = \Pr[X=x \mid Y=y]$
      \end{enumerate}

    \item Brownian motion
    \item Stationary processes (exchangeable)
    \item Gaussian processes
    \item L\'evy processes
  \end{enumerate}
