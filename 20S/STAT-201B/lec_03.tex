\lecture{3}{2020-01-28}{More on Markov chains}

A few more things about Markov chains which we can do with little cost,
with general state space $(S, \cS)$, discrete time $n = 0, 1, \ldots$.

General view of how to construct a sequence of RVs $(X_i)_{i \geq 0}$:
\begin{enumerate}
  \item Make $X_0 \sim \lambda$, $\Pr(X_0 \in A) = \lambda(A)$
    where $\lambda = \delta_x = \text{Dirac unit mass at $x$}$.
  \item Given $X_0 = x$, we have a Markov transition kernel
    \[
      P(x_0, \cdot) = \Pr(X_1 \in \cdot \mid X_0 = x_0)
    \]
    Construct $(X_0, X_1)$ using Fubini
    \[
      \bE g(X_0, X_1)
      = \int_S \lambda (dx_0) \int_S P(x_0, dx_1) g(x_0, x_1)
    \]
  \item Given $X_0 = x_0$ and $X_1 = x_1$, make $X_2$ according to
    \[
      P(x_0, x_1, \cdot) = \Pr(X_2 \in \cdot \mid X_0 = x_0, X_1 = x_1)
    \]
    We can do that! Nothing new required since
    \[
      X_0 \mapsto (X_0, X_1), \qquad \lambda \mapsto \lambda \otimes P~\text{on}~S \times S
    \]
    Now
    \[
      \bE g(X_0, X_1, X_2) = \underbrace{\int_S \lambda(dx_0) \int_S P(x_0, dx_1)}_{\int_{S \times S} (\lambda \otimes P)(dx_0, dx_1)}
      \int_S P(x_0, x_1, dx_2) g(x_0, x_1, x_2)
    \]
\end{enumerate}
if $S$ is nice, every stochastic process $(X_i)_{i \geq 0}$ is
``like'' (i.e. FDDs are equal in distribution) the above construction.

\begin{definition}
  Such a process is called \emph{Markov} iff it is equal in distribution
  to a process made with a kernel
  \[
    P(dx_{n+1} \mid x_0, \ldots, x_n)
    = P_n(dx_{n+1} \mid x_n)
  \]
  If $P_n \equiv P$, then the process is called \emph{homogeneous}.
\end{definition}

\begin{itemize}
  \item Not mmuch general theory of inhomogeneous
  \item We will always henceforth assume homogeneous
  \item Mostly, discrete $S$
  \item Techniques for general $S$
    \begin{itemize}
      \item Figure it out for finite/countable $S$
      \item Write the args so they wrook more generally
    \end{itemize}
  \item Note: if you upgrade $X_n = Y_n \coloneqq (X_0, \ldots, X_n)$
  \item Technical point: can only use Fubini to prescribe FDDs.
    From our schemme, we create consistent FDDs and are left with
    two options:
    \begin{enumerate}
      \item Make the chain explicitly as $X_n =$ some measurable
        function on $(\Omega, \cF, \Pr)$. For example,
        for IID uniforms take $X_n = F^{-1}(U_n)$ for an
        IID sequence $U_n$. Then you have RVs $(X_i)_{i \geq 0}$
        on $(\Omega, \cF, \Pr)$ which implies a law on
        $\prod_{i \geq 0} S$ given by
        \[
          \omega \mapsto (X_0(\omega), X_1(\omega), \ldots)
        \]
      \item If not, we need Kolmogorov consistency
        or better (Ionescu-Tulcea Theorem, which says the
        infinite process exists with no regularity on $(S, \cS)$).
    \end{enumerate}
\end{itemize}


Fix $(S, \cS)$ and Markov kernel $P$. For $f \geq 0$ bounded measurable
\[
  (Pf)(x)
  \coloneqq \int_S P(x, \cdot) f(\cdot)
  = \bE[ f(X_1) \mid X_0 = x]
\]
\textbf{Convention}: We write $\bE[f(X_1) \mid X_0 = x] = \bE_x f(X_1)$,
or more generally $\bE[f(X_1) \mid X_0 \sim \lambda] = \bE_\lambda f(X_1)$.

\begin{example}
  $\bE_\lambda f(X_0, X_1, X_2) = \int_S \lambda(dx) \bE_x f(x, X_1, X_2)$ by the Fubini theorem.
  This is an example with a regular conditional
  distribution (RCD) for $(X_0, X_1, X_2 \mid X_0 = x)$.

  More generally, anytime we see an expectation involving
  an integral we are really just doing conditioning. Most
  things in Markov chains are obvious by writing down
  Fubini.
\end{example}

A special case: $\bE_x f(X_1) = f(x)$ for all
$x \in S$ means that $(P - I) f = 0$ which means
that $(f(X_0), f(X_1))$ is a martingale
pair relative to $\sigma(X_0)$, $\sigma(X_0, X_1)$.
More generally, for any initial distribution $\lambda$ the collection $(f(X_i))_{i \geq 0}$
is
a martingale relative to $\cF_n = \sigma(X_i : i \leq n)$. For a martingale
\[
  \bE[f(X_{n+1} \mid X_0, \ldots, X_n] = g(X_n)
\]
by the time homogeneous markov property, we must
have $g = P f$. So if $P f = f$, then
$\bE[\cdot] = f(X_n)$. This is the definition
of a MG.

\begin{example}[Gambler's Ruin]
  \label{eg:gamblers-ruin}

  $0 \leq a \leq b$, Gambler starts with $a$ dollars
  and at each turn moves $+1$ with probability $p$
  and $-1$ with probability $q$. Continue until
  reach $0$ or $b$ and then stop.

  \begin{figure}[H]
    \centering
    \incfig{gamblers-ruin}
    \caption{Gambler's ruin}
    \label{fig:gamblers-ruin}
  \end{figure}

  We can write this as $S_0 = a$ and $S_n = a + \sum_i X_i$ where $X_i$ are $\pm 1$ with probabilities
  $p/q$. Define (the stopping time)
  \[
    T_{0b} = \min\{n \geq 1 : S_n \in \{0,b\}\}
  \]
  Our MC is then $(S_{n \land T_{0b}} : n \geq 0)$.
  Viewed as a MC, we have an absorbing chain and are interested
  in $\Pr_a[\text{absorbed at $b$}]$.

  Alternatively, viewed as a RW we are interested in
  \[
    \Pr_a[S_{T_{0b}} = b]
  \]
  By Wald's identity
  \[
    \bE[S_{T_{0b}} - a]
    = \bE[X] \bE[T_{0b}]
  \]
  For fair case, $p = q = 1/2$ and $\bE X = 0$.
  \begin{align*}
    (b-a) \Pr_a[\text{absorb at b}] + (0 - a) \Pr_a[\text{absorb at 0}] + \underbrace{??}_{0} &= 0 \\
    \Pr_a[\text{absorb at b}] + \Pr_a[\text{absorb at 0}] = 1
  \end{align*}
  How do we know it eventually hits and doesn't oscillate forever?
  Bounded by Geometric RV, so goiing to be finite with probability 1
  (could have use martingale convergence, but dont need).

  In the unfair case, we need a better martingale (Wald MG). Try an
  exponential martingale
  \[
    Z^{S_n}~\text{for suitable $Z$}
  \]
  To find $Z$, we would like
  \begin{align*}
    \bE[Z^{\underbrace{S_{n+1}}_{=S_n + X_{n+1}}} \mid S_0, \ldots, S_n] &= Z^{S_n} \\
    \iff \bE Z^{X_{n+1}} &= 1 \\
    \implies Z &= q/p
  \end{align*}
  Applying Wald's identity again, we solve the system
  \begin{align*}
    \left(\frac{q}{p} \right)^a &= \Pr_a[\text{abs at b}] (q/p)^b + \Pr_a[\text{abs at 0}](q/p)^0 \\
    1 &= \Pr_a[\cdot] + \Pr_a[\cdot]
  \end{align*}

  Within the Markov chain frammework, we say that we seek a harmonic function $h$
  such that $h = P h$ and $h(0) = 0$, $h(b) = 1$. Then
  \[
    \Pr_a[\text{hit b before 0}] = ??
  \]
  Notice $h_0(x) = (q/p)^x$ and $h_1(x) \equiv 1$ both solve $h = P h$,
  so finding the linear combination that satisfies boundary
  conditions yields
  \[
    h(x) = \frac{(q/p)^x - (q/p)^0}{(q/p)^b - (q/p)^0}
  \]
\end{example}


\begin{theorem}
  Suppose you have a MC with absorbing states (i.e.\ boundary states)
  $B \subset S$ and target states $A \subset B$.
  \[
    \Pr_a\left[X_n \in A~\text{for all large $n$}\right]
  \]
  Let
  \[
    h_A(x)
    = \Pr_x[\text{absorved in $A$}]
  \]
  Argue $h(x) = h_A(x)$ must solve $h = P h$ using a one-step analysis:
  \begin{align*}
    h(x) &= \sum_{y \in S} P(x,y) h(y)
  \end{align*}
  by conditioning on $X_1 = y$

    Notice $h_A(x) = \lim_n P^n \ind_A$ and
    \[
      (P^n \ind_A)(x) = \Pr_x[X_n \in A]
    \]
    Since $A$ is absorbing, $X_n \in A \implies X_{n+1} \in A$ so
    $\{X_n \in A\}$ is increasing, so $h_A = P h_A$.

    Also, $h_A = \ind_A$ on $B$ because $B$ is absorbing.
\end{theorem}

Since Dirichlet BVP has unique soution, we hope/expect that there is a
unique solution of $h = P h$, $h = \ind_A$ on $B$, and that in this
case $h = h_A$.
